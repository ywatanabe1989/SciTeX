\begin{abstract}
\pdfbookmark[1]{Abstract}{abstract}

Working memory (WM) is indispensable for numerous cognitive functions, although the underlying neural mechanisms have not been fully elucidated. While the hippocampus and sharp-wave ripple complexes (SWRs) -- brief, synchronous neural events within the hippocampus -- are acknowledged for their importance in memory consolidation and retrieval, their association with WM tasks remains unclear. The present study posits that hippocampal multiunit activity patterns, in concert with SWRs, manifest distinct dynamics during WM tasks. We analyzed a dataset comprising intracranial electroencephalogram recordings from the medial temporal lobe (MTL) of nine epilepsy patients engaged in an eight-second Sternberg task. Gaussian-process factor analysis was utilized to extract low-dimensional neural representations, or 'trajectories', within MTL regions during the WM task. Our results reveal that the hippocampus demonstrates the most significant fluctuation in neural trajectory relative to the entorhinal cortex and amygdala. Furthermore, trajectories' dissimilarity measured between encoding and retrieval phases is dependent on memory load. Notably, hippocampal trajectories fluctuate during the retrieval phase and exhibit task-dependent shifts between encoding and retrieval states, both during baseline and SWR events. This fluctuation transitions from encoding to retrieval states in the presence of SWRs. These findings reinforce the pivotal role of the hippocampus in WM tasks and suggest a new hypothesis: the hippocampus alters its functional state from encoding to retrieval during SWRs.

\end{abstract}