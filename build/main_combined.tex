\UseRawInputEncoding


%%%%%%%%%%%%%%%%%%%%%%%%%%%%%%%%%%%%%%%%%%%%%%%%%%%%%%%%%%%%%%%%%%%%%%%%%%%%%%%%
%% Settings
%%%%%%%%%%%%%%%%%%%%%%%%%%%%%%%%%%%%%%%%%%%%%%%%%%%%%%%%%%%%%%%%%%%%%%%%%%%%%%%%
%% Columns
\documentclass[final,3p,times,twocolumn]{elsarticle}
%% Use the options 1p,twocolumn; 3p; 3p,twocolumn; 5p; or 5p,twocolumn
%% for a journal layout:
%% \documentclass[final,1p,times]{elsarticle}
%% \documentclass[final,1p,times,twocolumn]{elsarticle}
%% \documentclass[final,3p,times]{elsarticle}
%% \documentclass[final,3p,times,twocolumn]{elsarticle}
%% \documentclass[final,5p,times]{elsarticle}
%% \documentclass[final,5p,times,twocolumn]{elsarticle}
%% \documentclass[preprint,review,12pt]{elsarticle}

%% Image width
\newlength{\imagewidth}
\newlength{\imagescale}
%% preamble
\usepackage[english]{babel}
\usepackage[table]{xcolor} % For coloring tables
\usepackage{booktabs} % For professional quality tables
\usepackage{colortbl} % For coloring cells in tables
\usepackage{amsmath, amssymb} % For mathematical symbols and environments
\usepackage{amsthm} % For theorem-like environments
\usepackage{lipsum} % just for sample text
\usepackage{natbib}
\usepackage{graphicx}
\usepackage{indentfirst}
\usepackage{bashful}
% for figures
\usepackage[margin=10pt,font=small,labelfont=bf,labelsep=endash]{caption}
\usepackage{graphicx}
\usepackage{calc}
% for tables
\usepackage{xlsx2csv}
\usepackage{csv2latex}
\usepackage[T1]{fontenc} % [REVISED]
\usepackage[utf8]{inputenc} % [REVISED]
\usepackage{hyperref}
\usepackage{accsupp}

%% Line numbers
\linespread{1.1}
% \linenumbers

% Use accsupp package to make line numbers non-selectable/non-copyable in PDF
\renewcommand\LineNumber{%
  \BeginAccSupp{method=escape,ActualText={}}%
  \thelinenumber\ %
  \EndAccSupp{}%
}

% Ensure that listings package is loaded and configured if you want to include code listings
\usepackage{listings}
\lstset{
  numbers=left,
  numberstyle=\tiny,
  stepnumber=1,
  numbersep=5pt,
  basicstyle=\ttfamily,
  frame=tb,
  framesep=5pt,
  framexleftmargin=15pt,
  backgroundcolor=\color{gray!10}
}

\makeatletter
\def\lst@PlaceNumber{%
  \llap{\normalfont
    \pdfliteral direct{/Span<</ActualText()>>BDC}%
    \lst@numberstyle{\thelstnumber}%
    \pdfliteral direct{EMC}%
    \kern\lst@numbersep%
  }%
}
\makeatother
%%%%%%%%%%%%%%%%%%%%%%%%%%%%%%%%%%%%%%%%%%%%%%%%%%%%%%%%%%%%%%%%%%%%%%%%%%%%%%%%
%% Journal Name
%%%%%%%%%%%%%%%%%%%%%%%%%%%%%%%%%%%%%%%%%%%%%%%%%%%%%%%%%%%%%%%%%%%%%%%%%%%%%%%%
\journal{Heliyon}

%%%%%%%%%%%%%%%%%%%%%%%%%%%%%%%%%%%%%%%%%%%%%%%%%%%%%%%%%%%%%%%%%%%%%%%%%%%%%%%%
%% Document Starts
%%%%%%%%%%%%%%%%%%%%%%%%%%%%%%%%%%%%%%%%%%%%%%%%%%%%%%%%%%%%%%%%%%%%%%%%%%%%%%%%
\begin{document}


%%%%%%%%%%%%%%%%%%%%%%%%%%%%%%%%%%%%%%%%%%%%%%%%%%%%%%%%%%%%%%%%%%%%%%%%%%%%%%%%
%% frontmatter
%%%%%%%%%%%%%%%%%%%%%%%%%%%%%%%%%%%%%%%%%%%%%%%%%%%%%%%%%%%%%%%%%%%%%%%%%%%%%%%%
\begin{frontmatter}
\begin{highlights}
\pdfbookmark[1]{Highlights}{highlights}
\item Neural trajectory in the hippocampus showed greater variations during a WM task compared to EC and amygdala regions.
\item The distance of neural trajectory in the hippocampus between encoding and retrieval states were memory-load dependent.
\item Hippocampal neural trajectory fluctuated between the encoding and retrieval states in a task-dependent manner
\item Hippocampal neural trajectory shifted from encoding to retrieval states during sharp-wave ripple event
\end{highlights}
\title{Hippocampal neural fluctuation between memory encoding and retrieval in a working memory task in humans:  An encoding-to-retrieval shift during sharp-wave ripples}

\author[1]{Yusuke Watanabe\corref{cor1}}
\author[2,3,4]{Yuji Ikegaya}
\author[1,5]{Takufumi Yanagisawa}

\address[1]{Institute for Advanced Cocreation studies, Osaka University, 2-2 Yamadaoka, Suita, 565-0871, Osaka, Japan}
\address[2]{Graduate School of Pharmaceutical Sciences, The University of Tokyo, 7-3-1 Hongo, Tokyo, 113-0033, Japan}
\address[3]{Institute for AI and Beyond, The University of Tokyo, 7-3-1 Hongo, Tokyo, 113-0033, Japan}
\address[4]{Center for Information and Neural Networks, National Institute of Information and Communications Technology, 1-4 Yamadaoka, Suita City, 565-0871, Osaka, Japan}
\address[5]{Department of Neurosurgery, Osaka University Graduate School of Medicine, 2-2 Yamadaoka, Osaka, 565-0871, Japan}

\cortext[cor1]{Corresponding author. Tel: +81-6-6879-3652}
% \end{title*}
%%Graphical abstract
%\pdfbookmark[1]{Graphical Abstract}{graphicalabstract}        
%\begin{graphicalabstract}
%\includegraphics{grabs}
%\end{graphicalabstract}

\begin{abstract}
\pdfbookmark[1]{Abstract}{abstract}
Working memory (WM) is critical for various cognitive functions, yet the underlying neural mechanisms remain obscure. Although the hippocampus and sharp-wave ripple complexes (SWRs) --- transient and synchronous neural bursts in the hippocampus --- are recognized for their roles in memory consolidation and retrieval, their relationship with WM tasks is poorly understood. This study hypothesizes that multiunit activity patterns in the human hippocampus, in conjunction with SWRs, exhibit distinct behaviors during WM tasks. To test this hypothesis, we employed a dataset consisting of intracranial electroencephalogram recordings from the medial temporal lobe (MTL) of nine epilepsy patients during an eight-second Sternberg test. Gaussian-process factor analysis was utilized to derive low-dimensional neural representations, or 'trajectories', specific to WM tasks within MTL areas. Our findings demonstrate that the hippocampus displays the most significant variations in neural trajectory compared to the entorhinal cortex and amygdala. Moreover, the distance of trajectories between encoding and retrieval phases was memory-load dependent. Notably, the hippocampal trajectory fluctuated between the encoding and retrieval stats in a task-dependent manner, with a shift from encoding to retrieval states during SWRs. These observations underscore the hippocampus's pivotal role in WM tasks and provide novel insights into the functional contributions of the hippocampus to WM demands.
\end{abstract}
% \pdfbookmark[1]{Keywords}{keywords}                
\begin{keyword}
working memory \sep WM \sep memory load \sep hippocampus \sep sharp-wave ripples \sep SWR \sep humans
\end{keyword}

\end{frontmatter}

%%%%%%%%%%%%%%%%%%%%%%%%%%%%%%%%%%%%%%%%%%%%%%%%%%%%%%%%%%%%%%%%%%%%%%%%%%%%%%%%
%% main
%%%%%%%%%%%%%%%%%%%%%%%%%%%%%%%%%%%%%%%%%%%%%%%%%%%%%%%%%%%%%%%%%%%%%%%%%%%%%%%%
%%%%%%%%%%%%%%%%%%%%%%%%%%%%%%%%%%%%%%%%%%%%%%%%%%%%%%%%%%%%%%%%%%%%%%%%%%%%%%%%
%% Introduction
%%%%%%%%%%%%%%%%%%%%%%%%%%%%%%%%%%%%%%%%%%%%%%%%%%%%%%%%%%%%%%%%%%%%%%%%%%%%%%%%
\section{Introduction}
Working memory (WM) is crucial in everyday life; however, its neural mechanism has yet to be fully elucidated. Specifically, the role of the hippocampus, an essential brain region guiding memory and spatial navigation, has been a topic of ongoing controversy \cite{scoville_loss_1957} \cite{squire_legacy_2009}  \cite{boran_persistent_2019} \cite{kaminski_persistently_2017} \cite{kornblith_persistent_2017} \cite{faraut_dataset_2018} \cite{borders_hippocampus_2022} \cite{li_functional_2023} \cite{dimakopoulos_information_2022}. Understanding the hippocampus' role in working memory is instrumental in deepening our knowledge of cognitive processes, ultimately aiding in developing cognitive training strategies and interventions.
\\
\indent
Among the hippocampal phenomena, a transient and synchronous oscillation called sharp-wave ripple (SWR) \cite{buzsaki_hippocampal_2015} is associated with various cognitive functions, including memory replay \cite{wilson_reactivation_1994} \cite{nadasdy_replay_1999} \cite{lee_memory_2002} \cite{diba_forward_2007} \cite{davidson_hippocampal_2009}, memory consolidation \cite{girardeau_selective_2009} \cite{ego-stengel_disruption_2010} \cite{fernandez-ruiz_long-duration_2019} \cite{kim_corticalhippocampal_2022}, memory recall \cite{wu_hippocampal_2017} \cite{norman_hippocampal_2019} \cite{norman_hippocampal_2021}, and neural plasticity \cite{behrens_induction_2005} \cite{norimoto_hippocampal_2018}. However, investigations into the effects of SWRs on working memory remain infrequent (\cite{jadhav_awake_2012} and limited to rodent models using navigation tasks, in which the precise timings of memory acquisition and recall are not separated.
\\
\indent
Hippocampal neurons may exhibit low-dimensional representations during WM tasks. For instance, the firing patterns of place cells \cite{okeefe_hippocampus_1971} \cite{okeefe_place_1976} \cite{ekstrom_cellular_2003} \cite{kjelstrup_finite_2008} \cite{harvey_intracellular_2009} in the hippocampus were found to be embedded within a dynamic, nonlinear 3D hyperbolic geometry while rodent navigating \cite{zhang_hippocampal_2022}. Furthermore, grid cells in the entorhinal cortex (EC) --- the primary gateway to the hippocampus \cite{naber_reciprocal_2001} \cite{van_strien_anatomy_2009} \cite{strange_functional_2014} --- exhibited toroidal topology during exploration \cite{gardner_toroidal_2022}. However, again, these experiments are limited to spatial navigation tasks in rodents so that the temporal resolution of WM tasks is constrained. Moreover, whther these findings are generalized to humans, especially other than navigation tasks, are not investigated yet.
\indent
\\
Given these backgrounds, in this study, we investigated the hypothesis that hippocampal neurons exhibit distinct representations in low-dimensional spaces as 'neural trajectory' during WM tasks, with a specific focus on SWR periods. To test this hypothesis, we utilized a dataset of patients performing an eight-second Sternbeug task with high temporal resolution (1 s for fixation, 2 s for encoding, 3 s for maintenance, and 2 s for retrieval) while their intrachranial electroencephalography signals (iEEG) in the medial temporal lobe (MTL) were recorded \cite{boran_dataset_2020}. To explore low-dimensional neural trajectories, we employed Gaussian-process factor analysis (GPFA) based on multiunit activities, a proven tool for the analysis of neural population dynamics \cite{yu_gaussian-process_2009}.
\label{sec:introduction}

%%%%%%%%%%%%%%%%%%%%%%%%%%%%%%%%%%%%%%%%%%%%%%%%%%%%%%%%%%%%%%%%%%%%%%%%%%%%%%%%
%% Methods
%%%%%%%%%%%%%%%%%%%%%%%%%%%%%%%%%%%%%%%%%%%%%%%%%%%%%%%%%%%%%%%%%%%%%%%%%%%%%%%%
\section{Methods}
\subsection{Dataset}
A publicly accessible dataset \cite{boran_dataset_2020} was employed, in which nine subjects performed a modified Sternberg task that consisted of fixation (1 s), encoding (2 s), maintenance (3 s), and retrieval (2 s) phases \cite{boran_dataset_2020}. During the encoding phase, participants were presented sets of either four, six, or eight alphabetical letters (set size). Subsequently, they were tasked with ascertaining whether a probe letter presented in the retrieval phase had been displayed (the correct choice for Match IN task) or not (the correct choice for Mismatch OUT task). iEEG signals were recorded at a sampling rate of 32 kHz, within the frequency range of 0.5--5,000 Hz, using depth electrodes implanted in the medial temporal regions. Specifically, electrodes were placed in the left and right hippocampal head (AHL and AHR), hippocampal body (PHL and PHR), entorhinal cortex (ECL and ECR), and amygdala (AL and AR) (Figure 1A and Table 1). Subsequently, iEEG signals were resampled at a rate of 2 kHz. Correlations were found among the experimental variables such as set size and correct rate (Figure S1). The timings of multiunit spikes were estimated by a spike sorting algorithm \cite{niediek_reliable_2016} by the Combinato package ((\url{https://github.com/jniediek/combinato})(Figure 1C).

\subsection{Calculation of neural trajectories using GPFA}
To calculate the neural trajectories (also referred to as factors; Figure 1D) in the hippocampus, EC, and amygdala (Figure 1D), GPFA \cite{yu_gaussian-process_2009}, was employed on the multiunit activity data for each session. GPFA was applied using the elephant package ((\url{https://elephant.readthedocs.io/en/latest/reference/gpfa.html}). The bin size was set as 50 ms, with no overlaps. Each factor was z-normalized across each session. From the trajectories, Euclidean distance from the origin ($O$ (0,0,0)) was calculated (Figure 1E).
\\
\indent
For every trajectory within a region (\textit{e.g.}, AHL), \textit{geometric medians} were determined by calculating the median coordinates of trajectory during the four phases: (\textit{i.e.}, $\mathrm{g_{F}}$ for fixation, $\mathrm{g_{E}}$ for encoding, $\mathrm{g_{M}}$ for maintenance, and $\mathrm{g_{R}}$ for retrieval phase) (Figure 1D). The optimal dimensionality for GPFA was determined as three via the elbow method utilizing the log-likelihood values using the threefold cross-validation approach (Figure 2B).

\subsection{Defining SWR candidates from hippocampal regions}
To identify potential SWR events within the hippocampus, we employed a detection method aligned with a consensus in this field \cite{liu_consensus_2022}. Specifically, local field potential (LFP) signals from a region of interest (ROI), such as AHL, were re-referenced by subtracting a control signal obtained by averaging signals from outside the ROI (\textit{e.g.}, AHR, PHL, PHR, ECL, ECR, AL, and AR) (see Figure 1A). The LFP signals were applied to a ripple-band filter (80--140 Hz) to isolate SWR candidates (SWR$^+$ candidates) (see Figure 1B). SWR detection was conducted using a published tool ((\url{https://github.com/Eden-Kramer-Lab/ripple_detection}) \cite{kay_hippocampal_2016}, with modifications such as an updated bandpass range of 80--140 Hz for humans from original 150--250 Hz range primarily for rodents.
\\
\indent
As control events for SWR$^+$ candidates, SWR$^-$ candidates were defined by shuffling the timestamps of SWR$^+$ candidates across all trials from all subjects. Finally, the SWR$^+$/SWR$^-$ candidates were visually inspected (see Figure 1).

\subsection{Defining SWRs from putative hippocampal CA1 regions}
SWRs were defined from SWR candidates as follows. First, putative CA1 regions were defined as follows. First, SWR$^+$/SWR$^-$ candidates in the hippocampus were embedded into a two-dimensional space based on their superimposed spike counts per unit using UMAP (uniform manifold approximation and projection) \cite{mcinnes_umap_2018} in a supervised fashion (Figure 4A). The silhouette score \cite{rousseeuw_silhouettes_1987}, a validation barometer for clustering, was calculated from clustered samples (Table 2). The hippocampal regions with silhouette scores greater than 0.6 on average across sessions $75^{th}$ percentile) (Figure 4B) were defined as putative CA1 regions, identifying five eletrode positions from five patients (Table 3).
\\
\indent
As a second step, SWR$^+$/SWR$^-$ candidates in putative CA1 regions were defined as SWR$^+$/SWR$^-$ (no longer candidates). The duration and ripple band peak amplitude of detected SWRs were calculated (Figure 4C \& E). SWR$^+$/SWR$^-$ were visually inspected as shown in Figure 1. The duration and ripple band peak amplitudes of detected SWRs were quantified (Figure 4C \& E). Each SWR was split into pre-SWR (= event at $-800$ to $-300$ ms from SWR center), mid-SWR (= event at $-250$ to $+250$ ms from SWR center), and post-SWR (= event at $+300$ to $+800$ ms from SWR center).

\subsection{Statistical evaluation}
The Brunner--Munzel test and the Kruskal-Wallis test were executed using the scipy package in Python \cite{virtanen_scipy_2020}. A correlational analysis was undertaken through the determination of the rank of the observed correlation coefficient in the associated set-size-shuffled surrogate, using a custom Python script. Additionally, the bootstrap test was carried out by utilizing an internally developed Python script.
\label{sec:methods}
%%%%%%%%%%%%%%%%%%%%%%%%%%%%%%%%%%%%%%%%%%%%%%%%%%%%%%%%%%%%%%%%%%%%%%%%%%%%%%%%
%% Results
%%%%%%%%%%%%%%%%%%%%%%%%%%%%%%%%%%%%%%%%%%%%%%%%%%%%%%%%%%%%%%%%%%%%%%%%%%%%%%%%
\section{Results}
\subsection{iEEG recording and neural trajectory in MTL regions during a Sternberg task}
We employed a publicly available dataset \cite{boran_dataset_2020} for this analysis. This dataset recorded LFP signals (Figure 1A) within the MTL regions (Table 1) during a modified Sternberg task. SWR$^+$ candidates were identified from LFP signals passed through the ripple band (Figure 1B) within all hippocampal regions (refer to Methods), while SWR$^-$ candidates were designated at identical timestamps of the SWR$^+$ candidates but shuffling them across different trials (Figure 1). The multiunit spikes (Figure 1C) were included in the dataset, being established using a spike sorting algorithm \cite{niediek_reliable_2016}. Using the 50-ms binned multiunit activity without overlaps, we employed GPFA \cite{yu_gaussian-process_2009} to determine the neural trajectory (or factors) of the MTL regions by session and region (Figure 1D). Each factor was z-normalized by session and region (for example, session \#2 in AHL of subject \#1). Subsequently, the Euclidean distance from the origin ($O$) was calculated (Figure 1E).

\subsection{Hippocampal neural trajectory correlated with a WM task}
In Figure 2A, the median neural trajectories of 50 trials are depicted as points within the three major factor space. The optimal embedding dimension for the GPFA model was determined to be three using the elbow method (Figure 2B). The trajectory distance from the origin ($O$) for the hippocampus was larger compared to the EC and amygdala (Figure C \& D).\footnote{Hippocampus: Distance = 1.11 [1.01], median [IQR], \textit{n} = 195,681 timepoints; EC: Distance = 0.94 [1.10], median [IQR], \textit{n} = 133,761 timepoints; Amygdala: Distance = 0.78 [0.88], median [IQR], \textit{n} = 165,281 timepoints.}
\\
\indent
Similarly, the distance among geometric medians of the four phases were calculated: \textit{i.e.}, $\mathrm{\lVert g_{F}g_{E} \rVert}$) for the distance between fixation and encoding states. Again, the hippocampus showed larger distances among phases compared to both the EC and amygdala. \footnote{Hippocampus: Distance = 0.60 [0.70], median [IQR], \textit{n} = 8,772 combinations; EC: Distance = 0.28 [0.52], median [IQR], \textit{n} = 5,017 combinations (\textit{p} $<$ 0.01; Brunner--Munzel test); Amygdala: Distance = 0.24 [0.42], median [IQR], \textit{n} = 7,466 combinations (\textit{p} $<$ 0.01; Brunner--Munzel test).}

\subsection{Memory load-dependent neural trajectory distance between the encoding and retrieval states in the hippocampus}
Correct rate of trials and set size (the number of alphabetical letters to encode) were negatively correlated (Figure 3A). \footnote{Correct rate: set size four (0.99 \textpm 0.11, mean \textpm SD; \textit{n} = 333 trials) vs. set size six (0.93 \textpm 0.26; \textit{n} = 278 trials; \textit{p} $<$ 0.001, Brunner--Munzel test with Bonferroni correction) and set size eight (0.87 \textpm 0.34; \textit{n} = 275 trials; \textit{p} $<$ 0.05; Brunner--Munzel test with Bonferroni correction). Overall, \textit{p} $<$ 0.001 for Kruskal--Wallis test; correlation coefficient = - 0.20, \textit{p} $<$ 0.001.} Similarly, response time and set size were positively correlated (Figure 3B).\footnote{Response time: set size four (1.26 \textpm 0.45 s; \textit{n} = 333 trials) vs. set size six (1.53 \textpm 0.91 s; \textit{n} = 278 trials) and set size eight (1.66 \textpm 0.80 s; \textit{n} = 275 trials). All comparisons \textit{p} $<$ 0.001, Brunner--Munzel test with Bonferroni correction; \textit{p} $<$ 0.001 for Kruskal--Wallis test; correlation coefficient = 0.22, \textit{p} $<$ 0.001}
\\
\indent
Set size and the trajectory distance between the encoding and retrieval phases ($\mathrm{log_{10}\lVert g_{E}g_{R} \rVert}$) were positively correlated (Figure 3C).\footnote{Correlation between set size and $\mathrm{log_{10}(\lVert g_{E}g_{R} \rVert}$): correlation coefficient = 0.05, \textit{p} $<$ 0.001. Specific values: $\mathrm{\lVert g_{E}g_{R} \rVert}$ = 0.54 [0.70] for set size four trials, \textit{n} = 447; $\mathrm{\lVert g_{E}g_{R} \rVert}$ = 0.58 [0.66] for set size six trials, \textit{n} = 381; $\mathrm{\lVert g_{E}g_{R} \rVert}$ = 0.61 [0.63] for set size eight trials, \textit{n} = 395.}. However, no significant correlations were found between set size and distance among other phase combinations (Figures 3D \& S2).

\subsection{Detection of hippocampal SWR from putative CA1 regions}
Under the aim to improve the precision of recording sites and the detection of SWRs, we identified electrodes in putative CA1 regions of the hippocampus based on observing distinct multiunit spike patterns during SWR events. For each session and hippocampal region, SWR$^+$/SWR$^-$ candidates were embedded into a two-dimensional space via UMAP (Figure 4A).\footnote{For illustrative purposes, consider the AHL in session \#1 of subject \#1.} We calculated the silhouette score as a measure of clustering quality (Figure 4B \& Table 2). Recording sites with an average silhouette score across sessions exceeding 0.6 were designated as putative CA1 regions\footnote{The identified regions were: AHL of subject \#1, AHR of subject \#3, PHL of subject \#4, AHL of subject \#6, and AHR of subject \#9.}  (Tables 2 \& 3). Consequently, four regions out of the five putative CA1 areas had not been identified as seizure onset zones, while one had been designated (Table 1).
\\
\indent
Subsequently, SWR$^+$/SWR$^-$ candidates within these putative CA1 regions were labeled SWR$^+$ and SWR$^-$, respectively\footnote{Definitions lead to equal counts for both categories: SWR$^+$ (\textit{n} = 1,170) and SWR$^-$ (\textit{n} = 1,170).}  (Table 3). Both SWR$^+$ and SWR$^-$ exhibited an identical duration\footnote{Definitions lead to equal duration for both categories: SWR$^+$ (93.0 [65.4] ms) and SWR$^-$ (93.0 [65.4] ms).}  (Figure 4C) due to their definitions, following a log-distribution profile. A marked increase in SWR$^+$ incidence was detected during the initial 400 ms from the onset of the retrieval phase \footnote{SWR$^+$ increased against the bootstrap sample; 95th percentile = 0.42 [Hz]; \textit{p} $<$ 0.05.}  (Figure 4D). Additionally, the peak ripple band amplitude for SWR$^+$ exceeded that of SWR$^-$ and followed a log-normal distribution (Figure 4E).\footnote{SWR$^+$ (3.05 [0.85] SD of baseline, median [IQR]; \textit{n} = 1,170) vs. SWR$^-$ (2.37 [0.33] SD of baseline, median [IQR]; \textit{n} = 1,170; \textit{p} $<$ 0.001; Brunner--Munzel test).}.

\subsection{Transient neural trajectory change in the hippocampus during SWR}
The distances of trajectory from the origin ($O$) during SWR events in both the encoding and retrieval phases (\textit{\textit{i.e.}}, eSWR$^+$, eSWR$^-$, rSWR$^+$, and rSWR$^-$) were calculated (Figure 5A). Given the pronounced peak in them, we categorized each SWR into three stages: pre-SWR, mid-SWR, and post-SWR. Subsequently, the distances from $O$ during these SWR periods are represented as $\mathrm{\lVert \text{pre-eSWR}^+ \rVert}$, $\mathrm{\lVert \text{mid-eSWR}^+ \rVert}$, and so on.
\\
\indent
$\mathrm{\lVert \text{mid-eSWR}^+ \rVert}$
\footnote{1.25 [1.30], median [IQR], \textit{n} = 1,281, in Match IN task; 1.12 [1.35], median [IQR], \textit{n} = 1,163, in Mismatch OUT task}
was larger than $\mathrm{\lVert \text{pre-eSWR}^+ \rVert}$
\footnote{1.08 [1.07], median [IQR], \textit{n} = 1,149, in Match IN task; 0.90 [1.12], median [IQR], \textit{n} = 1,088, in Mismatch OUT task}
(Figure 5B). Similarly, $\mathrm{\lVert \text{mid-rSWR}^+ \rVert}$
\footnote{1.32 [1.24], median [IQR], \textit{n} = 935, in Match IN task; 1.15 [1.26], median [IQR], \textit{n} = 891, in Mismatch OUT task}
was larger than $\mathrm{\lVert \text{pre-rSWR}^+ \rVert}$.
\footnote{1.19 [0.96], median [IQR], \textit{n} = 673, in Match IN task; 0.94 [0.88], median [IQR], \textit{n} = 664, in Mismatch OUT task}

\subsection{Visualization of hippocampal neural trajectory during SWR in two-dimensional spaces}
Based on our observations of neural trajectory 'jump' during SWR (Figure 5), we visualized the trajectories of pre-, mid-, and post-SWR events during the encoding and retrieval phases (Figure 6), the distance of which was memory-load dependendent (Figure 3).
\\
\indent
To achieve the visualization in two dimension spaces, the peri-SWR trajectories were aligned linearly by positioning $\mathrm{g_{E}}$ at the origin (0, 0) and placing $\mathrm{g_{R}}$ at ($\mathrm{\lVert g_{E}g_{R} \rVert}$, 0). These aligned trajectories were rotated around the x axis for visualization purposes. Importantly, distances and angles in the original three-dimensional spaces are preserved in these two-dimensional ones.
\\
\indent
The scatter plot in these two-dimensional spaces illustrates characteristic distributions of peri-SWR trajectories based on phases and task types. For instance, longer distance of mid-eSWR$^+$, compaired to pre-eSWR$^-$, from $\mathrm{g_{E}}$ can be observed (Figure 6B), consistent with our earlier findings (Figure 5).

\subsection{Fluctuations of hippocampal neural trajectories between encoding and retrieval states}
Subsequently, we checked trajectory directions based on $\overrightarrow{\mathrm{g_{E}g_{R}}}$. SWR directions were defined by neural trajectory at $-250$ ms and $+250$ ms from their center (\textit{i.e.}, $\overrightarrow{\mathrm{eSWR^+}}$).
\\
\indent
$\overrightarrow{\mathrm{eSWR^+}}\cdot\overrightarrow{\mathrm{rSWR^+}}$ showed a bias towards $+1$ in Match IN task (Figure 7A) but towards $-1$ in Mismatch OUT task (Figure 7B). These tendencies were also observed in $\overrightarrow{\mathrm{eSWR^-}}\cdot\overrightarrow{\mathrm{rSWR^-}}$ (Figure 7C--F).
\\
\indent
Moreover, $\overrightarrow{\mathrm{rSWR^+}}\cdot\overrightarrow{\mathrm{g_{E}g_{R}}}$ showed a biphasic distribution in Match In task (Figure 7A) in contrast to a monophasic distribution in Mismatch In task (Figure 7B).
\\
\indent
Both in Match IN and Mismatch OUT tasks, by taking the differences between the distribution of $\overrightarrow{\mathrm{rSWR^+}}\cdot\overrightarrow{\mathrm{g_{E}g_{R}}}$ (Figure 7A \& B) and that of $\overrightarrow{\mathrm{rSWR^-}}\cdot\overrightarrow{\mathrm{g_{E}g_{R}}}$ (Figure 7C \& D), the effect of SWR was revealed as a shift from $\mathrm{g_{E}}$ to $\mathrm{g_{R}}$ (Figure 7E \& F).
\label{sec:results}

%%%%%%%%%%%%%%%%%%%%%%%%%%%%%%%%%%%%%%%%%%%%%%%%%%%%%%%%%%%%%%%%%%%%%%%%%%%%%%%%
%% Discussion
%%%%%%%%%%%%%%%%%%%%%%%%%%%%%%%%%%%%%%%%%%%%%%%%%%%%%%%%%%%%%%%%%%%%%%%%%%%%%%%%
\section{Discussion}
This study hypothesized that hippocampal multiunit activity exhibits distinct representations, or trajectories, in low-dimensional spaces during a WM task in humans, particularly during SWR periods. First, we projected the multiunit activity from MTL regions during a Sternberg task onto three-dimensional spaces by GPFA (Figure 1D--E and Figure 2A). The distance of trajectory among WM phases was larger in the hippocampus than the EC and amygdala (Figure 2E), showing more dynamical neural activity in the hippocampus during the WM task. Additionally, the distance trajectory between the encoding and retrieval phases in the hippocampus ($\mathrm{\lVert g_{F}g_{E} \rVert}$) was positively correlated with memory load (Figure 3C \& D), indicating it as a reflection of WM processing. Furthermore, the trajectory exhibited a transient 'jump' during SWRs (Figure 5). Finally, the hippocampal neural trajectory fluctuated between encoding and retrieval states, with a shift from encoding to retrieval during SWR events (Figure 7). In sum, these results demonstrated the hippocampal neural behavior in a WM task in humans.
% For visualization purposes, peri-SWR trajectories were embedded into two-dimensional spaces, showing task-specifi
\\
\indent
First, we found that the distance of the trajectory among the four phases of the WM task ($\mathrm{\lVert g_{F}g_{E} \rVert}$, $\mathrm{\lVert g_{F}g_{M} \rVert}$, $\mathrm{\lVert g_{F}g_{R} \rVert}$, $\mathrm{\lVert g_{E}g_{M} \rVert}$, $\mathrm{\lVert g_{E}g_{R} \rVert}$, and $\mathrm{\lVert g_{M}g_{R} \rVert}$) was longer in the hippocampus compared to the EC and amygdala, even considering the distance from $O$ ($\mathrm{\lVert g_{F} \rVert}$, $\mathrm{\lVert g_{E} \rVert}$, $\mathrm{\lVert g_{M} \rVert}$, and $\mathrm{\lVert g_{R} \rVert}$) in those regions (Figure 2C--E). These results indicate hippocampal participation in the WM task, which is partially supported by previous findings of hippocampal persistent firing in the maintenance phase \cite{boran_persistent_2019} \cite{kaminski_persistently_2017} \cite{kornblith_persistent_2017} \cite{faraut_dataset_2018}. However, by applying GPFA to multiunit activity during the 1-s level resolution of WM task, we revealed that the trajectory in low dimensional space displayed memory-load dependent distance between the encoding and retrieval phase, or $\mathrm{\lVert g_{E}g_{R} \rVert}$ (Figure 3). Overall, these results provide evidence that the hippocampus is linked to WM.
\\
\indent
The validity of confining our analysis to putative CA1 regions (Figure 4) is supported by several factors. First, this targeted approach stems from well-established observations that SWRs are time-locked to synchronous spike bursts of interneurons and pyramidal neurons \cite{buzsaki_two-stage_1989} \cite{quyen_cell_2008} \cite{royer_control_2012} \cite{hajos_input-output_2013}, possibly around 50 $\mu$m radius of the recording site \cite{schomburg_spiking_2012}. Additionally, the incidence of detected SWR increased at 0--400 ms from the onset of the retrieval phase (Figure 4D), which is consistent with previous reports showing increased hippocampal SWR occurrence before spontaneous verbal recall \cite{norman_hippocampal_2019} \cite{norman_hippocampal_2021}. Moreover, the log-normal distributions of SWR duration and ripple band peak amplitude observed in this study (Figure 4C \& E) align with the consensus in this field \cite{liu_consensus_2022}. Therefore, our approach would have increased precision of SWR detection. One limitation is that the increase in trajectory distance from $O$ during SWR (Figure 5) would have been biased because of the channel selection; however, this possibility is not critical for our major findings.
\\
\indent
Interestingly, the trajectory directions in the retrieval phase oscillated between the encoding and retrieval states (Figure 7C \& D). In addition, the balance of such fluctuation was shifted from the encoding to retrieval state during SWR (Figure 7 E \& F; red rectangle-dotted line). These results are again consistent with previous reports suggesting SWR's role in memory recall \cite{norman_hippocampal_2019} \cite{norman_hippocampal_2021}. Therefore, our results will provide new aspects of the hippocampal representations: (i) neural fluctuations between encoding and retrieval states during a WM task and (ii) SWR as a switching representation from encoding to retrieval state.
\\
\indent
Moreover, our study reveals WM-task type specific directions of SWR trajectory (Figure 7). To illustrate, $\overrightarrow{\mathrm{eSWR^+}}$ and $\overrightarrow{\mathrm{rSWR^+}}$ were closely aligned in Match IN task (Figure 7A; pink dotted line), while they are directed to the adverse direction in Mismatch OUT task (Figure 7B; pink dotted line). These result might be explained by the memory engram theory \cite{liu_optogenetic_2012}. Match In task exposed subjects to once-seen letter, while Mismatch OUT task make them encounter with a novel letter. These results suggest that SWR is highly related to working cognitive processess in humans.
\\
\indent
In conclusion, our study has demonstrated that hippocampal activity fluctuates between encoding and retrieval states during baseline periods and exhibits a significant transition from the encoding state to the retrieval state during SWRs. These findings suggest the applicability of the two-stage memory formation model \cite{buzsaki_two-stage_1989} to working memory tasks on a timescale of seconds.

\label{sec:discussion}


%%%%%%%%%%%%%%%%%%%%%%%%%%%%%%%%%%%%%%%%%%%%%%%%%%%%%%%%%%%%%%%%%%%%%%%%%%%%%%%%
%% Reference Styles
%%%%%%%%%%%%%%%%%%%%%%%%%%%%%%%%%%%%%%%%%%%%%%%%%%%%%%%%%%%%%%%%%%%%%%%%%%%%%%%%
\pdfbookmark[1]{References}{references}
\bibliography{./build/refs}
% Note Re-compile is required

%% Numbering Style (sorted)
\bibliographystyle{elsarticle-num}

% Author Style
% \bibliographystyle{plainnat}
% use \citet{}

%% Numbering Style (not-sorted) 
% \bibliographystyle{plainnat}
% use \cite{}




%%%%%%%%%%%%%%%%%%%%%%%%%%%%%%%%%%%%%%%%%%%%%%%%%%%%%%%%%%%%%%%%%%%%%%%%%%%%%%%%
%% Additional Information
%%%%%%%%%%%%%%%%%%%%%%%%%%%%%%%%%%%%%%%%%%%%%%%%%%%%%%%%%%%%%%%%%%%%%%%%%%%%%%%%
%%%%%%%%%%%%%%%%%%%%%%%%%%%%%%%%%%%%%%%%%%%%%%%%%%%%%%%%%%%%%%%%%%%%%%%%%%%%%%%%
%% Additional Information
%%%%%%%%%%%%%%%%%%%%%%%%%%%%%%%%%%%%%%%%%%%%%%%%%%%%%%%%%%%%%%%%%%%%%%%%%%%%%%%%
\pdfbookmark[1]{Additional Information}{additional_information}

\pdfbookmark[2]{Contributors}{contributors}                    
\section*{Contributors}
Y.W. and T.Y. conceptualized the study; Y.W. performed the data analysis; Y.W. and T.Y. wrote the original draft; and all authors reviewed the final manuscript.
\label{contributors}

\pdfbookmark[2]{Acknowledgments}{acknowledgments}                    
\section*{Acknowledgments}
This research was funded by a grant from the Exploratory Research for Advanced Technology (JPMJER1801).
\label{acknowledgments}

\pdfbookmark[2]{Declaration of Interests}{declaration_of_interest}                    
\section*{Declaration of Interests}
The authors declare that they have no competing interests.
\label{declaration of interests}

\pdfbookmark[2]{Data and code availability}{data_and_code_availability}                    
\section*{Data and code availability}
The data is available on G-Node (\url{https://doi.gin.g-node.org/10.12751/g-node.d76994/}). The source code is available on GitHub (\url{https://github.com/yanagisawa-lab/hippocampal-neural-fluctuation-during-a-WM-task-in-humans}).
\label{data and code availability}

\pdfbookmark[2]{Inclusion and Diversity Statement}{inclusion_and_diversity_statement}        
\section*{Inclusion and Diversity Statement}
We support inclusive, diverse, and equitable conduct of research.
\label{inclusion and diversity statement}

\pdfbookmark[2]{Declaration of Generative AI in Scientific Writing}{declaration_of_generative_ai}
\section*{Declaration of Generative AI in Scientific Writing}
The authors employed ChatGPT, provided by OpenAI, for enhancing the manuscript's English language quality. After incorporating the suggested improvements, the authors meticulously revised the content. Ultimate responsibility for the final content of this publication rests entirely with the authors.
\label{declaration of generative ai in scientific writing}


%%%%%%%%%%%%%%%%%%%%%%%%%%%%%%%%%%%%%%%%%%%%%%%%%%%%%%%%%%%%%%%%%%%%%%%%%%%%%%%%
%% Tables
%%%%%%%%%%%%%%%%%%%%%%%%%%%%%%%%%%%%%%%%%%%%%%%%%%%%%%%%%%%%%%%%%%%%%%%%%%%%%%%%
\clearpage
\section*{Tables}
\label{tables}
\pdfbookmark[1]{Tables}{tables}
\pdfbookmark[2]{ID 01}{id_01}
\begin{table*}[ht]
\centering
\rowcolors{3}{gray!25}{white}
\rowcolor{white}
\begin{tabular}{|l|l|l|l|l|l|l|l|l|l|l|l|}
\hline
&&Hipp. head&&Hipp. Body&&EC&&Amy.&&&\\
\hline
Subject ID&\# of &&&&&&&&&&\\
\hline
sessions&AHL&AHR&PHL&PHR&ECL&ECR&AL&AR&&SOZ&\\
\hline
\#1&4&o&x&o&o&o&x&o&x&&AHR, LR\\
\hline
\#2&7&o&o&o&o&o&o&o&o&&AHR, PHR\\
\hline
\#3&3&o&o&o&o&o&o&o&x&&AHL, PHL\\
\hline
\#4&2&o&o&o&o&o&o&o&o&&AHL, AHR, PHL, PHR\\
\hline
\#5&3&o&x&x&o&x&x&o&x&&DRR\\
\hline
\#6&6&o&o&o&o&o&o&o&o&&AHL, PHL, ECL, AL\\
\hline
\#7&4&o&o&o&o&o&o&o&o&&AHR, PHR\\
\hline
\#8&5&o&o&o&o&o&o&o&o&&ECR\\
\hline
\#9&2&o&o&o&o&o&o&o&o&&ECR, AR\\
\hline
\bottomrule
\end{tabular}
\caption{\textbf{
Electrode positions of the dataset
}
\smallskip
\\
The electrode positions and the seizure onset zones. Regions marked as "o" were available, but those marked as "x" (\textit{navy}) were not available in the dataset. Abbreviations: AHL, left hippocampal head; AHR, right hippocampal head; PHL, left hippocampal body; PHR, right hippocampal body; ECL, left entorhinal cortex; ECR, right entorhinal cortex; AL, left amygdala; AR, right amygdala, SOZ: seizure onset zone.
}
% width=1\textwidth\label{tab:01}
\end{table*}
\clearpage
\pdfbookmark[2]{ID 02}{id_02}
\begin{table*}[ht]
\centering
\rowcolors{3}{gray!25}{white}
\rowcolor{white}
\begin{tabular}{|l|l|l|l|l|}
\hline
&Hipp. head&&Hipp. body&\\
\hline
Subject&AHL&AHR&PHL&PHR\\
\hline
\#1&0.60 ± 0.14&n.a.&n.a.&0.1 ± 0\\
\hline
\#2&0.21 ± 0.16&0.17 ± 0.21&0.18 ± 0.22&0.20 ± 0.15\\
\hline
\#3&0.40 ± 0.42&0.83 ± 0.12&n.a.&n.a.\\
\hline
\#4&0.10 ± 0.00&0.10 ± 0.00&0.90 ± 0.00&0.10 ± 0.14\\
\hline
\#5&n.a.&n.a.&n.a.&n.a.\\
\hline
\#6&0.63 ± 0.06&n.a.&n.a.&0.27 ± 0.06\\
\hline
\#7&0.10 ± 0.00&0.35 ± 0.35&0.37 ± 0.47&0.10 ± 0.00\\
\hline
\#8&0.13 ± 0.10&n.a.&0.28 ± 0.49&n.a.\\
\hline
\#9&n.a.&0.85 ± 0.07&0.15 ± 0.07&n.a.\\
\hline
\bottomrule
\end{tabular}
\caption{\textbf{
The silhouette score of UMAP clustering between $SWR^+$ candidates and $SWR^-$ candidates
}
\smallskip
\\
The silhouette scores (mean ± SD for sessions by subject) of UMAP clustering on SWR+ candidates and SWR− candidates (Figure 4A) were based on their underlying multiunit spike patterns (mean values were 0.205 [0.285], median [IQR]; Figure 4B).
}
% width=1\textwidth\label{tab:02}
\end{table*}
\clearpage
\pdfbookmark[2]{ID 03}{id_03}
\begin{table*}[ht]
\centering
\rowcolors{3}{gray!25}{white}
\rowcolor{white}
\begin{tabular}{|l|l|l|l|l|l|}
\hline
Subject ID&\# of sessions&\# of trials&ROI&\# of SWRs&SWR incidence [Hz]\\
\hline
\#1&2&100&AHL&274&0.34\\
\hline
\#3&2&97&AHR&325&0.42\\
\hline
\#4&2&99&PHL&202&0.26\\
\hline
\#6&2&100&AHL&297&0.37\\
\hline
\#9&2&97&AHR&72&0.092784\\
\hline
&Total = 10&Total = 493&&Total = 1,170&0.30 ± 0.13\\
\hline
(mean ± SD)&&&&&\\
\hline
&&&&&\\
\hline
\bottomrule
\end{tabular}
\caption{\textbf{
The number of defined SWR events
}
\smallskip
\\
The table summarizes the statistics of putative CA1 regions and SWRs. Only the first two sessions (sessions #1 and #2) from each subject were utilized to reduce the sampling bias.
}
% width=1\textwidth\label{tab:03}
\end{table*}


%%%%%%%%%%%%%%%%%%%%%%%%%%%%%%%%%%%%%%%%%%%%%%%%%%%%%%%%%%%%%%%%%%%%%%%%%%%%%%%%
%% Figures
%%%%%%%%%%%%%%%%%%%%%%%%%%%%%%%%%%%%%%%%%%%%%%%%%%%%%%%%%%%%%%%%%%%%%%%%%%%%%%%%
\clearpage
\section*{Figures}
\label{figures}
\pdfbookmark[1]{Figures}{figures}
        \clearpage
        \begin{figure*}[ht]
            \pdfbookmark[2]{ID 01}{figure_id_01}
        	\centering
            \includegraphics[width=1\textwidth]{./media/figures/.png/Figure_ID_01.png}
        	\caption{\textbf{
Local field potential (LFP), multiunit activity, and neural trajectory of the hippocampus during a modified Sternberg task
}
\smallskip
\\
\textbf{\textit{A.}} Wideband LFP Trace: Displaying the intracranial encephalogram (iEEG) data for subject \#6, session \#2, trial \#5, located in the left hippocampal head. This trace showcases the varying phases of the modified Sternberg working memory task, including fixation (\textit{gray}, 1 s), encoding (\textit{blue}, 2 s), maintenance (\textit{green}, 3 s), and retrieval (\textit{red}, 2 s). \textbf{\textit{B.}} Ripple Band LFP: This presents the LFP trace in the ripple frequency band from the same trial and session. \textbf{\textit{C.}} Multiunit Spike Raster: A plot illustrating multiunit spikes recorded during the mentioned trial. \textbf{\textit{D.}} Neural Trajectory Factors: Depicting the first three factors of the hippocampal trajectory for the trial, determined by Gaussian-process factor analysis on spike counts with 50-ms bins. The dot circles signify the geometric median coordinates for each of the four task phases. \textbf{\textit{E.}} Trajectory Distance Analysis: Shows the distance of the initial three trajectory factors from the origin ($O$). Note that highlighted rectangles overlaying the figure indicate the timings for SWR$^+$ candidates (in \textit{purple}) and SWR$^-$ candidates (in \textit{yellow}).
}
% width=1\textwidth
        	\label{fig:01}
        \end{figure*}
        \clearpage
        \begin{figure*}[ht]
            \pdfbookmark[2]{ID 02}{figure_id_02}
        	\centering
            \includegraphics[width=0.5\textwidth]{./media/figures/.png/Figure_ID_02.png}
        	\caption{\textbf{
State-dependent hippocampal neural trajectory
}
\smallskip
\\
\textbf{\textit{A.}} Neural Trajectory Visualization: This three-dimensional plot represents the neural trajectory of the left hippocampus derived from the Gaussian-process factor analysis (GPFA). Smaller points indicate coordinates of 50-ms neural trajectory bins within a session (median of 50 trials). Larger points with \textit{black} edges represent geometric medians for each phase of the Sternberg working memory task, distinguished by colors for fixation (\textit{gray}), encoding (\textit{blue}), maintenance (\textit{green}), and retrieval (\textit{red}). \textbf{\textit{B.}}  GPFA Model Log-likelihoods: The graph showcases the log-likelihood predictions of GPFA models concerning the number of dimensions. Notably, the optimal dimension was identified as three using the elbow method. \textbf{\textit{C.}}  Distance Analysis from $O$rigin: The plot represents the trajectory distance from the origin ($O$) for the hippocampus (Hipp.), entorhinal cortex (EC), and amygdala (Amy.), plotted against the time since probe onset. \textbf{\textit{D.}}  Trajectory Distance in Medial Temporal Regions: The box plots display the trajectory distances from $O$ for each region, with the hippocampus showing the greatest distance, followed by the EC and the Amygdala. \textbf{\textit{E.}}  Inter-phase Trajectory Distances: This visualizes the distance variations between trial trajectory geometric medians of task phases. The distances in the hippocampus were observed to be more pronounced compared to the EC or Amygdala.
}
% width=0.5\textwidth
        	\label{fig:02}
        \end{figure*}
        \clearpage
        \begin{figure*}[ht]
            \pdfbookmark[2]{ID 03}{figure_id_03}
        	\centering
            \includegraphics[width=1\textwidth]{./media/figures/.png/Figure_ID_03.png}
        	\caption{\textbf{
Influence of memory load on encoding and retrieval states in the hippocampus
}
\smallskip
\\
\textbf{\textit{A.}} Relationship between set size (number of letters encoded) and correct rate in the WM task. Notably, a negative correlation was observed (coefficient = $-$ 0.20, ***\textit{p} $<$ 0.001), analyzed using set-size-shuffled surrogate. Significance was determined using Kruskal--Wallis test, followed by the Brunner--Munzel test with Bonferroni correction. \textbf{\textit{B.}}  Association between set size and response time post-probe initiation. A positive correlation was detected (coefficient = 0.23, ***\textit{p} $<$ 0.001), analyzed using set-size-shuffled surrogate. Statistical tests included the Kruskal--Wallis and post-hoc Brunner--Munzel tests with Bonferroni correction. \textbf{\textit{C.}}  Analysis of set size against the distance between geometric medians during encoding and retrieval phases (represented as $\lVert \mathrm{g_{E}g_{R}} \rVert$). A correlation of 0.05 was noted for set size and $\mathrm{log_{10}{\lVert g_{E}g_{R} \rVert}}$), analyzed using set-size-shuffled surrogate. \textbf{\textit{D.}}  Experimentally observed correlations between set size and various parameters are illustrated by \textit{red} dots. The \textit{gray} violin plots contrast these findings with set-size-shuffled surrogate data (\textit{n} = 1,000), emphasizing the significant observed correlation coefficients (***\textit{p} $<$ 0.001). Abbreviations: $\mathrm{g_{F}}$, $\mathrm{g_{E}}$, $\mathrm{g_{M}}$, $\mathrm{g_{R}}$ represent the geometric median of trajectories during fixation, encoding, maintenance, and retrieval phases, respectively.
}
% width=1\textwidth
        	\label{fig:03}
        \end{figure*}
        \clearpage
        \begin{figure*}[ht]
            \pdfbookmark[2]{ID 04}{figure_id_04}
        	\centering
            \includegraphics[width=1\textwidth]{./media/figures/.png/Figure_ID_04.png}
        	\caption{\textbf{
SWR detection in putative CA1 regions
}
\smallskip
\\
\textbf{\textit{A.}} Two-dimensional UMAP (uniform manifold approximation and projection)\cite{mcinnes_umap_2018} representation of unit activities during SWR$^+$ candidates (\textit{purple}; potential SWR events) and SWR$^-$ candidates (\textit{yellow}; controls for SWR$^+$ candidates). \textbf{\textit{B.}}  Cumulative density plot illustrating silhouette scores for different hippocampal regions (refer to Table 2). Regions with silhouette scores exceeding 0.60 ($75^{th}$ percentile) are considered as putative CA1 regions. Within these regions, SWR$^+$ and SWR$^-$ candidates are identified and categorized as SWR$^+$ (\textit{n} = 1,170) and SWR$^-$ (\textit{n} = 1,170), respectively. \textbf{\textit{C.}}  Overlapping duration distributions [ms] for SWR$^+$ (\textit{purple}; 93.0 [65.4], median [IQR]) and SWR$^-$ (\textit{yellow}; 93.0 [65.4], median [IQR]). The identical overlap results from their definition criteria. \textbf{\textit{D.}}  Depiction of ripple incidence [Hz] for both SWR$^+$ (\textit{purple}) and SWR$^-$ (\textit{yellow}) relative to time from the probe, represented as mean \textpm 95\% confidence interval. However, the 95\% confidence interval might not be readily discernible due to its narrow range. Note the significant elevation in SWR incidence between 0--400 ms post-probe, surpassing the 95th percentile of bootstrap samples (0.421 [Hz], *\textit{p} $<$ 0.05). \textbf{\textit{E.}}  Histogram showcasing ripple band peak amplitude distributions for SWR$^-$ (\textit{yellow}; 2.37 [0.33], median [IQR]) versus SWR$^+$ (\textit{purple}; 3.05 [0.85], median [IQR]). A notable difference is present between the two, confirmed with ***\textit{p} $<$ 0.001 using the Brunner--Munzel test.
}
% width=1\textwidth
        	\label{fig:04}
        \end{figure*}
        \clearpage
        \begin{figure*}[ht]
            \pdfbookmark[2]{ID 05}{figure_id_05}
        	\centering
            \includegraphics[width=1\textwidth]{./media/figures/.png/Figure_ID_05.png}
        	\caption{\textbf{
Transient neural trajectory change during SWR
}
\smallskip
\\
\textbf{\textit{A.}} Distance from the origin ($O$) of the peri-sharp-wave-ripple trajectory (mean \textpm 95\% confidence interval; however, the interval may not be easily visible due to the narrow range.).  \textbf{\textit{B.}}  The distance from the origin during the mid-SWR$^+$ period was found to be longer compared to the corresponding pre-SWR$^+$ period (*\textit{p} $<$ 0.05, **\textit{p} $<$ 0.01, ***\textit{p} $<$ 0.001; Brunner--Munzel test). Abbreviations: SWR, sharp-wave ripple events; eSWR, SWR during the encoding phase; rSWR, SWR during the retrieval phase, SWR$^+$, SWR event; SWR$^-$ control events for SWR$^+$; pre-SWR, mid-SWR, or post-SWR, the time interval from $-800$ to $-250$ ms, from $-250$ to $+250$ ms, or from $+250$ to $+800$ ms from the center of SWR.
}
% width=1\textwidth
        	\label{fig:05}
        \end{figure*}
        \clearpage
        \begin{figure*}[ht]
            \pdfbookmark[2]{ID 06}{figure_id_06}
        	\centering
            \includegraphics[width=1\textwidth]{./media/figures/.png/Figure_ID_06.png}
        	\caption{\textbf{
Coordinates of neural trajectory during sharp-wave ripple aligned by encoding and retrieval states.
}
\smallskip
\\
\textbf{\textit{A.}} Hippocampal neural trajectories during pre- (\textit{gray}), mid- (\textit{yellow}), and post-SWR$^-$ (\textit{black}) in Match IN (\textit{left}) and Mismatch OUT task (\textit{right}). \textbf{\textit{B.}} The equivalents for SWR$^+$ instead of SWR$^-$, though mid-SWR$^+$ is depicted with \textit{purple}. All data points underwent adjustments and rotations to fit a two-dimensional representation, positioning $\mathrm{g_{E}}$ at (0, 0) and $\mathrm{g_{R}}$ at ($\lVert \mathrm{g_{E}g_{R}} \rVert$, 0). The $\lVert \mathrm{g_{E}g_{R}} \rVert$ metric varies across sessions, and its median \textpm IQR (with medians approximately 0.2) is presented on the x-axes. Note: some intervals might be challenging to discern because of their narrow range. In this two-dimensional depiction, both the distances and angles preserve their relationships as in the original three-dimensional space. Abbreviations: SWR, sharp-wave ripple events; eSWR, SWR during the encoding phase; rSWR, SWR during the retrieval phase, SWR$^+$, SWR event; SWR$^-$ control events for SWR$^+$; pre-SWR, mid-SWR, or post-SWR, the time interval from $-800$ to $-250$ ms, from $-250$ to $+250$ ms, or from $+250$ to $+800$ ms from the center of SWR.
}
% width=1\textwidth
        	\label{fig:06}
        \end{figure*}
        \clearpage
        \begin{figure*}[ht]
            \pdfbookmark[2]{ID 07}{figure_id_07}
        	\centering
            \includegraphics[width=0.5\textwidth]{./media/figures/.png/Figure_ID_07.png}
        	\caption{\textbf{
Analysis of Neural Trajectory Directions during SWR between Encoding, and Retrieval States.
}
\smallskip
\\
\textbf{\textit{A--B}} Kernel density estimation (KDE) distribution of $\protect\overrightarrow{{\mathrm{eSWR^+}}} \cdot \protect\overrightarrow{{\mathrm{rSWR^+}}}$ (\textit{pink circles}), $\protect\overrightarrow{{\mathrm{eSWR^+}}} \cdot \protect\overrightarrow{{\mathrm{g_{E}g_{R}}}}$ (\textit{blue triangles}), and $\protect\overrightarrow{{\mathrm{rSWR^+}}} \cdot \protect\overrightarrow{{\mathrm{g_{E}g_{R}}}}$ (\textit{red rectangles}) in Match In (\textbf{\textit{A}}) and Mismatch OUT task (\textbf{\textit{B}}). \textbf{\textit{C--D.}} The corresponding distributions of $\mathrm{SWR^-}$ in response to those of $\mathrm{SWR^+}$ in \textbf{\textit{A--B}}. \textbf{\textit{E--F.}} The differences in distributions, highlighting the SWR components (\textbf{\textit{E}} = \textbf{\textit{C}} $-$ \textbf{\textit{A}}; \textbf{\textit{F}} = \textbf{\textit{B}} $-$ \textbf{\textit{D}}). Note the inverse directionality between $\protect\overrightarrow{{\mathrm{eSWR^+}}}$ and $\protect\overrightarrow{{\mathrm{rSWR^+}}}$ only in Mismatch OUT task (\textit{pink circles} in \textbf{\textit{E--F}}). Additionally, the shifts from the retrieval to encoding states were observed for SWR components both in Match IN and Mismatch OUT tasks (\textit{red rectangles} in \textbf{\textit{E--F}}).
}
% width=0.5\textwidth
        	\label{fig:07}
        \end{figure*}


%%%%%%%%%%%%%%%%%%%%%%%%%%%%%%%%%%%%%%%%%%%%%%%%%%%%%%%%%%%%%%%%%%%%%%%%%%%%%%%%
%% Appendices
%%%%%%%%%%%%%%%%%%%%%%%%%%%%%%%%%%%%%%%%%%%%%%%%%%%%%%%%%%%%%%%%%%%%%%%%%%%%%%%%
%% \pdfbookmark[1]{Appendices}{appendices}                    
%% \appendix
%% \section{}
%% \label{}


%%%%%%%%%%%%%%%%%%%%%%%%%%%%%%%%%%%%%%%%%%%%%%%%%%%%%%%%%%%%%%%%%%%%%%%%%%%%%%%%
%% END
%%%%%%%%%%%%%%%%%%%%%%%%%%%%%%%%%%%%%%%%%%%%%%%%%%%%%%%%%%%%%%%%%%%%%%%%%%%%%%%%

\end{document}
