\section{Introduction}
Working memory (WM) plays an essential role in everyday activities, yet comprehensive understanding of its neural mechanism, including the hippocampus's involvement in WM processing, remains elusive. The hippocampus, a central region for memory, is the focus of continuous research \cite{scoville_loss_1957} \cite{squire_legacy_2009}  \cite{boran_persistent_2019} \cite{kaminski_persistently_2017} \cite{kornblith_persistent_2017} \cite{faraut_dataset_2018} \cite{borders_hippocampus_2022} \cite{li_functional_2023} \cite{dimakopoulos_information_2022}. Gaining insight into the role of the hippocampus in working memory is key to enhancing our understanding of cognitive processes and could potentially improve cognitive capabilities.
\\
\indent
Current evidence suggests that sharp-wave ripples (SWRs), a form of transient, synchronized oscillation, is linked with several cognitive functions. These functions include memory replay \cite{wilson_reactivation_1994} \cite{nadasdy_replay_1999} \cite{lee_memory_2002} \cite{diba_forward_2007} \cite{davidson_hippocampal_2009}, memory consolidation \cite{girardeau_selective_2009} \cite{ego-stengel_disruption_2010} \cite{fernandez-ruiz_long-duration_2019} \cite{kim_corticalhippocampal_2022}, memory recall \cite{wu_hippocampal_2017} \cite{norman_hippocampal_2019} \cite{norman_hippocampal_2021}, and neural plasticity \cite{behrens_induction_2005} \cite{norimoto_hippocampal_2018}. These ties suggest that SWR serve as a fundamental computational mechanism of hippocampal processing, contributing to the performance of working memory.
\\
\indent
Recent studies show that low-dimensional representations in hippocampal neurons can account for WM task performance. In particular, the firing patterns of place cells \cite{okeefe_hippocampus_1971} \cite{okeefe_place_1976} \cite{ekstrom_cellular_2003} \cite{kjelstrup_finite_2008} \cite{harvey_intracellular_2009}, found in the hippocampus, were noted within dynamic, nonlinear three-dimensional hyperbolic spaces in rats \cite{zhang_hippocampal_2022}. Also, grid cells in the entorhinal cortex (EC)—the key conduit to the hippocampus \cite{naber_reciprocal_2001} \cite{van_strien_anatomy_2009} \cite{strange_functional_2014}—demonstrated toroidal geometry during exploration in rats \cite{gardner_toroidal_2022}.
\\
\indent
However, these studies primarily concentrate on spatial navigation in rodents and consequently, feature limitations. For instance, the temporal resolution of navigation tasks is insufficient because the timing of memory acquisition and recall has not been clearly defined. As a result, there are few studies examining the influence of SWRs on WM performance \cite{jadhav_awake_2012}. Furthermore, the existence of noise in signals recorded during rodent movement complicates the detection of SWRs \cite{Watanabe_2021}. Hence, human research is needed to elucidate the relationship between SWRs and WM tasks.
\\
\indent
Taking these factors into account, this study explores the hypothesis that during SWR periods, hippocampal neurons exhibit distinct neural trajectories (NTs) in low-dimensional space in response to WM tasks in humans. To substantiate this hypothesis, we used a dataset of patients performing an eight-second Sternberg task (one second for fixation, two seconds for encoding, three seconds for maintenance, and two seconds for retrieval) with high temporal resolution. For these patients, intracranial electroencephalography (iEEG) signals within the medial temporal lobe (MTL) were recorded \cite{boran_dataset_2020}. To investigate low-dimensional NTs, we employed Gaussian-process factor analysis (GPFA), a proven method for analyzing neural population dynamics \cite{yu_gaussian-process_2009}.
\label{sec:introduction}