%%%%%%%%%%%%%%%%%%%%%%%%%%%%%%%%%%%%%%%%%%%%%%%%%%%%%%%%%%%%%%%%%%%%%%%%%%%%%%%% 
%% Introduction
%%%%%%%%%%%%%%%%%%%%%%%%%%%%%%%%%%%%%%%%%%%%%%%%%%%%%%%%%%%%%%%%%%%%%%%%%%%%%%%% 
\section{Introduction}
Working memory (WM) plays a crucial role in everyday life, yet its underlying neural mechanisms remain to be fully elucidated. Particularly, the function of the hippocampus, a vital brain region contributing to memory, commands ongoing investigation \cite{scoville_loss_1957,squire_legacy_2009,boran_persistent_2019,kaminski_persistently_2017,kornblith_persistent_2017,faraut_dataset_2018,borders_hippocampus_2022,li_functional_2023,dimakopoulos_information_2022}. Gaining insight into the role of the hippocampus in working memory fosters a deeper understanding of cognitive processes and facilitates the development of cognitive training strategies and interventions.
\\
\indent
Transient and synchronous oscillations, referred to as sharp wave ripples (SWR), are known to be associated with various cognitive functions, including memory replay \cite{wilson_reactivation_1994,nadasdy_replay_1999,lee_memory_2002,davidson_hippocampal_2009}, memory consolidation \cite{girardeau_selective_2009,ego-stengel_disruption_2010,fernandez-ruiz_long-duration_2019,kim_corticalhippocampal_2022}, memory recall \cite{wu_hippocampal_2017,norman_hippocampal_2019,norman_hippocampal_2021}, and neural plasticity \cite{behrens_induction_2005,norimoto_hippocampal_2018}. Therefore, SWR might constitute a fundamental aspect of processing in the hippocampus and contribute to working memory performance. However, studies investigating the effects of SWRs on working memory remain scarce \cite{jadhav_awake_2012}, and are predominantly limited to rodent models using navigation tasks, in which the exact timings of memory acquisition and recall are not distinguished.
\\
\indent
Further, it has been discovered that hippocampal neurons exhibit low-dimensional representations during WM tasks. For instance, the firing patterns of place cells \cite{okeefe_hippocampus_1971,okeefe_place_1976,ekstrom_cellular_2003,kjelstrup_finite_2008,harvey_intracellular_2009,royer_control_2012} in the hippocampus are embedded within a dynamic, nonlinear three-dimensional hyperbolic geometry in rodents \cite{zhang_hippocampal_2022}. Moreover, grid cells in the entorhinal cortex (EC) --- the primary gateway to the hippocampus \cite{naber_reciprocal_2001,van_strien_anatomy_2009,strange_functional_2014} --- displayed toroidal topology during exploration \cite{gardner_toroidal_2022}. Nevertheless, these experiments are again constrained to spatial navigation tasks in rodents, limited in temporal resolution for WM tasks. Additionally, it has yet to be investigated whether these findings can be generalized to humans and tasks beyond navigation.
\\
\indent
In light of these considerations, this study explores the hypothesis that hippocampal neurons exhibit distinct representations in low-dimensional spaces as a 'neural trajectory' during WM tasks, with a particular emphasis on SWR periods. To test this hypothesis, we utilized a dataset of patients performing an eight-second Sternberg task (with high temporal resolution: 1 s for fixation, 2 s for encoding, 3 s for maintenance, and 2 s for retrieval) while their intracranial electroencephalography signals (iEEG) in the medial temporal lobe (MTL) were being recorded \cite{boran_dataset_2020}. We employed Gaussian-process factor analysis (GPFA) based on multiunit activities to explore low-dimensional neural trajectories, a known tool for analyzing neural population dynamics \cite{yu_gaussian-process_2009}.
\label{sec:introduction}