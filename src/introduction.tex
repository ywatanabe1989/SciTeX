Working memory (WM) is vitally important in daily activities and continues to be an active research area, specifically its neural basis. The hippocampus, known to be essential for memory, consistently remains at the center of this inquiry \cite{scoville_loss_1957} \cite{squire_legacy_2009}  \cite{boran_persistent_2019} \cite{kaminski_persistently_2017} \cite{kornblith_persistent_2017} \cite{faraut_dataset_2018} \cite{borders_hippocampus_2022} \cite{li_functional_2023} \cite{dimakopoulos_information_2022}. Insights into the hippocampus's role in working memory are crucial for advancing our understanding of cognitive processes and thereby promoting developments in cognitive training and interventions.
\\
\indent
Existing evidence proposes that a temporary, synchronized oscillation, known as sharp-wave ripple (SWR) \cite{buzsaki_hippocampal_2015}, is associated with several cognitive functions, such as memory replay \cite{wilson_reactivation_1994} \cite{nadasdy_replay_1999} \cite{lee_memory_2002} \cite{diba_forward_2007} \cite{davidson_hippocampal_2009}, memory consolidation \cite{girardeau_selective_2009} \cite{ego-stengel_disruption_2010} \cite{fernandez-ruiz_long-duration_2019} \cite{kim_corticalhippocampal_2022}, memory recall \cite{wu_hippocampal_2017} \cite{norman_hippocampal_2019} \cite{norman_hippocampal_2021}, along with neural plasticity \cite{behrens_induction_2005} \cite{norimoto_hippocampal_2018}. These indications suggest that SWR might be an essential part of hippocampal processing, thus contributing to working memory performance. Yet, studies examining the effect of SWRs on working memory are limited \cite{jadhav_awake_2012}, with most research predominantly focusing on rodent models engaged in navigation tasks where the timing of memory acquisition and recall is not explicitly outlined.
\\
\indent
Recent studies have illustrated that hippocampal neurons demonstrate low-dimensional representations during WM tasks. Specifically, the firing patterns of place cells \cite{okeefe_hippocampus_1971} \cite{okeefe_place_1976} \cite{ekstrom_cellular_2003} \cite{kjelstrup_finite_2008} \cite{harvey_intracellular_2009}, located in the hippocampus, have been shown to exist within a dynamic, nonlinear three-dimensional hyperbolic geometry in rodents \cite{zhang_hippocampal_2022}. Grid cells in the entorhinal cortex (EC)—the primary pathway to the hippocampus \cite{naber_reciprocal_2001} \cite{van_strien_anatomy_2009} \cite{strange_functional_2014}—showed a toroidal topology during exploration \cite{gardner_toroidal_2022}. However, these investigations have been confined to spatial navigation tasks in rodents, thus constraining the temporal resolution of WM tasks. The implication of these findings for human subjects and their extrapolation beyond navigation tasks is yet to be confirmed.
\\
\indent
Considering these factors, this study seeks to support the hypothesis that hippocampal neurons represent low-dimensional spaces distinctively, referenced here as 'neural trajectory,' particularly during SWR periods in WM tasks. To assess this proposition, we utilized a dataset of patients executing an eight-second Sternberg task featuring high temporal resolution (1 s for fixation, 2 s for encoding, 3 s for upkeep, and 2 s for retrieval) while their intracranial electroencephalography signals (iEEG) within the medial temporal lobe (MTL) were being recorded \cite{boran_dataset_2020}. To explore low-dimensional neural trajectories, we employed Gaussian-process factor analysis (GPFA), an acclaimed strategy for analyzing neural population dynamics \cite{yu_gaussian-process_2009}.
\label{sec:introduction}