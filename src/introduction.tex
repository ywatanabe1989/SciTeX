\section{Introduction}
Working memory (WM) is vital for multiple daily tasks, yet our understanding of the underlying neural mechanisms remains incomplete. The hippocampus, a critical region for memory in the brain, merits ongoing investigations \cite{scoville_loss_1957,squire_legacy_2009,boran_persistent_2019,kaminski_persistently_2017,kornblith_persistent_2017,faraut_dataset_2018,borders_hippocampus_2022,li_functional_2023,dimakopoulos_information_2022}. Enhancing our understanding of the hippocampus's role in working memory could improve insights into cognitive processes and foster the progression of cognitive training strategies and interventions. 

\indent
Sharp wave ripples (SWR), generated by the hippocampus, are short-lived and synchronous oscillations tied to cognitive functions such as memory replay \cite{wilson_reactivation_1994,nadasdy_replay_1999,lee_memory_2002,davidson_hippocampal_2009}, memory consolidation \cite{girardeau_selective_2009,ego-stengel_disruption_2010,fernandez-ruiz_long-duration_2019,kim_corticalhippocampal_2022}, memory recall \cite{wu_hippocampal_2017,norman_hippocampal_2019,norman_hippocampal_2021}, and neural plasticity \cite{behrens_induction_2005,norimoto_hippocampal_2018}. Thus, SWRs could have a significant role in hippocampal processing, potentially affecting working memory performance. However, the exploration of SWRs' impact on working memory is scarce \cite{jadhav_awake_2012}, primarily focusing on rodent models executing navigation tasks without clear distinction between memory recall and acquisition timings.

\indent
Moreover, hippocampal neurons are reported to demonstrate low-dimensional representations during WM tasks. Specifically, the firing patterns of hippocampal place cells \cite{okeefe_hippocampus_1971,okeefe_place_1976,ekstrom_cellular_2003,kjelstrup_finite_2008,harvey_intracellular_2009,royer_control_2012} are reported to align with a dynamic, nonlinear three-dimensional hyperbolic geometry in rodents \cite{zhang_hippocampal_2022}. Similarly, grid cells in the entorhinal cortex (EC)—the primary route to the hippocampus \cite{naber_reciprocal_2001,van_strien_anatomy_2009,strange_functional_2014}—present a toroidal topology during exploration \cite{gardner_toroidal_2022}. Regrettably, these studies largely relate to spatial navigation tasks in rodents and provide limited temporal resolution for WM tasks. They also fail to establish whether these findings could be applicable to humans or to tasks outside of navigation.

\indent
Given the aforementioned points, this study seeks to test the hypothesis that hippocampal neurons exhibit unique low-dimensional representations, or 'neural trajectories', during WM tasks, specifically during SWR occurrences. In investigating this, we utilized a dataset from a patient performing an eight-second Sternberg task (which provides high temporal resolution: 1 s for fixation, 2 s for encoding, 3 s for maintenance, and 2 s for retrieval) while recording their medial temporal lobe (MTL) intracranial electroencephalography signals (iEEG) \cite{boran_dataset_2020}. We implemented Gaussian-process factor analysis (GPFA) on multichannel unit activity to observe low-dimensional neural trajectories, a proven method for analyzing neural population dynamics \cite{yu_gaussian-process_2009}.
\label{sec:introduction}