%%%%%%%%%%%%%%%%%%%%%%%%%%%%%%%%%%%%%%%%%%%%%%%%%%%%%%%%%%%%%%%%%%%%%%%%%%%%%%%%
%% Discussion
%%%%%%%%%%%%%%%%%%%%%%%%%%%%%%%%%%%%%%%%%%%%%%%%%%%%%%%%%%%%%%%%%%%%%%%%%%%%%%%%
\section{Discussion}
This study hypothesized that hippocampal multiunit activity exhibits distinct representations, or trajectories, in low-dimensional spaces during a WM task in humans, particularly during SWR periods. First, we projected the multiunit activity from MTL regions during a Sternberg task onto three-dimensional spaces by GPFA (Figure 1D--E and Figure 2A). The distance of trajectory among WM phases was larger in the hippocampus than the EC and amygdala (Figure 2E), showing more dynamical neural activity in the hippocampus during the WM task. Additionally, the distance trajectory between the encoding and retrieval phases in the hippocampus ($\mathrm{\lVert g_{F}g_{E} \rVert}$) was positively correlated with memory load (Figure 3C \& D), indicating it as a reflection of WM processing. Furthermore, the trajectory exhibited a transient 'jump' during SWRs (Figure 5). Finally, the hippocampal neural trajectory fluctuated between encoding and retrieval states, with a shift from encoding to retrieval during SWR events (Figure 7). In sum, these results demonstrated the hippocampal neural behavior in a WM task in humans.
% For visualization purposes, peri-SWR trajectories were embedded into two-dimensional spaces, showing task-specifi
\\
\indent
First, we found that the distance of the trajectory among the four phases of the WM task ($\mathrm{\lVert g_{F}g_{E} \rVert}$, $\mathrm{\lVert g_{F}g_{M} \rVert}$, $\mathrm{\lVert g_{F}g_{R} \rVert}$, $\mathrm{\lVert g_{E}g_{M} \rVert}$, $\mathrm{\lVert g_{E}g_{R} \rVert}$, and $\mathrm{\lVert g_{M}g_{R} \rVert}$) was longer in the hippocampus compared to the EC and amygdala, even considering the distance from $O$ ($\mathrm{\lVert g_{F} \rVert}$, $\mathrm{\lVert g_{E} \rVert}$, $\mathrm{\lVert g_{M} \rVert}$, and $\mathrm{\lVert g_{R} \rVert}$) in those regions (Figure 2C--E). These results indicate hippocampal participation in the WM task, which is partially supported by previous findings of hippocampal persistent firing in the maintenance phase \cite{boran_persistent_2019} \cite{kaminski_persistently_2017} \cite{kornblith_persistent_2017} \cite{faraut_dataset_2018}. However, by applying GPFA to multiunit activity during the 1-s level resolution of WM task, we revealed that the trajectory in low dimensional space displayed memory-load dependent distance between the encoding and retrieval phase, or $\mathrm{\lVert g_{E}g_{R} \rVert}$ (Figure 3). Overall, these results provide evidence that the hippocampus is linked to WM.
\\
\indent
The validity of confining our analysis to putative CA1 regions (Figure 4) is supported by several factors. First, this targeted approach stems from well-established observations that SWRs are time-locked to synchronous spike bursts of interneurons and pyramidal neurons \cite{buzsaki_two-stage_1989} \cite{quyen_cell_2008} \cite{royer_control_2012} \cite{hajos_input-output_2013}, possibly around 50 $\mu$m radius of the recording site \cite{schomburg_spiking_2012}. Additionally, the incidence of detected SWR increased at 0--400 ms from the onset of the retrieval phase (Figure 4D), which is consistent with previous reports showing increased hippocampal SWR occurrence before spontaneous verbal recall \cite{norman_hippocampal_2019} \cite{norman_hippocampal_2021}. Moreover, the log-normal distributions of SWR duration and ripple band peak amplitude observed in this study (Figure 4C \& E) align with the consensus in this field \cite{liu_consensus_2022}. Therefore, our approach would have increased precision of SWR detection. One limitation is that the increase in trajectory distance from $O$ during SWR (Figure 5) would have been biased because of the channel selection; however, this possibility is not critical for our major findings.
\\
\indent
Interestingly, the trajectory directions in the retrieval phase oscillated between the encoding and retrieval states (Figure 7C \& D). In addition, the balance of such fluctuation was shifted from the encoding to retrieval state during SWR (Figure 7 E \& F; red rectangle-dotted line). These results are again consistent with previous reports suggesting SWR's role in memory recall \cite{norman_hippocampal_2019} \cite{norman_hippocampal_2021}. Therefore, our results will provide new aspects of the hippocampal representations: (i) neural fluctuations between encoding and retrieval states during a WM task and (ii) SWR as a switching representation from encoding to retrieval state.
\\
\indent
Moreover, our study reveals WM-task type specific directions of SWR trajectory (Figure 7). To illustrate, $\overrightarrow{\mathrm{eSWR^+}}$ and $\overrightarrow{\mathrm{rSWR^+}}$ were closely aligned in Match IN task (Figure 7A; pink dotted line), while they are directed to the adverse direction in Mismatch OUT task (Figure 7B; pink dotted line). These result might be explained by the memory engram theory \cite{liu_optogenetic_2012}. Match In task exposed subjects to once-seen letter, while Mismatch OUT task make them encounter with a novel letter. These results suggest that SWR is highly related to working cognitive processess in humans.
\\
\indent
In conclusion, our study has demonstrated that hippocampal activity fluctuates between encoding and retrieval states during baseline periods and exhibits a significant transition from the encoding state to the retrieval state during SWRs. These findings suggest the applicability of the two-stage memory formation model \cite{buzsaki_two-stage_1989} to working memory tasks on a timescale of seconds.

\label{sec:discussion}
