\begin{abstract}
\pdfbookmark[1]{Abstract}{abstract}
Working memory (WM) plays a vital role in many cognitive functions, though the complex neural mechanisms that support its operation are not yet fully understood. Although the importance of the hippocampus and sharp-wave ripple complexes (SWRs) -- fast, concurrent neural activities within the hippocampus -- are acknowledged in memory consolidation and retrieval, their participation in WM tasks remains somewhat ambiguous. Our study hypothesizes that multiunit activity patterns within the hippocampus cooperate synergistically with SWRs, showcasing distinctive dynamism during WM tasks. To investigate this, we carried out a thorough analysis of a dataset sourced from intracranial electroencephalogram recordings taken from the medial temporal lobe (MTL) of nine epilepsy patients performing an eight-second Sternberg task. Employing Gaussian-process factor analysis, we distinguished low-dimensional neural representations, or 'trajectories,' within the MTL regions during the WM task. Our findings showed that neural trajectories exhibited greater variation in the hippocampus compared to the entorhinal cortex and amygdala. Additionally, differences found in the trajectories between encoding and retrieval phases depended on memory load. Notably, hippocampal trajectories oscillated during the retrieval phase, demonstrating task-dependent shifts between encoding and retrieval states, along with baseline and SWR events. These oscillations transitioned from encoding to retrieval states in sync with SWRs. Consequently, our findings underscore the critical function of the hippocampus in performing WM tasks, and suggest a compelling hypothesis for further investigation: the operational state of the hippocampus shifts from encoding to retrieval during SWRs.
\end{abstract}