\begin{abstract}
\pdfbookmark[1]{Abstract}{abstract}
Working memory (WM) plays a critical role in many cognitive functions, but the intricate neural mechanisms that support its operation remain elusive. Specifically, while the hippocampus and sharp-wave ripple complexes (SWRs) -- brief, synchronous neural oscillation observed in the hippocampus -- are recognized for their roles in memory consolidation and retrieval, their involvement in WM tasks has not yet been defined. Current research suggests that during WM tasks, multiunit activity patterns in the hippocampus display distinctive dynamics, particularly during SWR periods. This study analyzed a dataset derived from intracranial electroencephalogram recordings made in the medial temporal lobe (MTL) of nine individuals with epilepsy during an eight-second Sternberg task. We applied Gaussian-process factor analysis to determine low-dimensional neural representations, or 'trajectories,' within the MTL regions while performing the WM task. The results indicate significant variations in the hippocampus' neural trajectories compared to those in the entorhinal cortex and amygdala. Additionally, the distance of the trajectory between the encoding and retrieval phases was dependent on memory load. Importantly, hippocampal trajectories during the retrieval phase demonstrated variations between encoding and retrieval stages based on task type, particularly showing shifts from encoding to retrieval states during SWRs. These findings underline the hippocampus's essential function in performing WM tasks and propose an intriguing hypothesis for future research: the functional state of the hippocampus transition from encoding to retrieval during SWRs.
\end{abstract}