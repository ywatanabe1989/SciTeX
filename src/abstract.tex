\begin{abstract}
\pdfbookmark[1]{Abstract}{abstract}
Working memory (WM) plays a critical role in diverse cognitive functions, yet its neural mechanisms remain largely unelucidated. An emerging area of focus is the role of the hippocampus and sharp wave-ripple complexes (SWRs) – fleeting, synchronized neural events in the hippocampus – in memory consolidation and retrieval, although their connection to WM tasks remains unclear. Recent research suggests that multiunit activity patterns in the hippocampus may function concurrently with SWRs, displaying unique dynamics during WM tasks. We conducted an analysis of an electroencephalogram dataset from the medial temporal lobe (MTL) in nine epilepsy patients during an eight-second Sternberg task. Low-dimensional neural representations, or 'trajectories', within the MTL were isolated using Gaussian-process factor analysis while performing the WM task. The results reveal significant differences in the neural trajectories in the hippocampus in comparison to the entorhinal cortex and the amygdala. Furthermore, the variance in trajectories between the encoding and retrieval phases seems to be memory load-dependent. Interestingly, hippocampal trajectories vary during the retrieval phase, indicating task-dependent shifts between encoding and retrieval states which occur during both baseline and SWR events. These shifts from encoding to retrieval states are synchronized with the occurrence of SWRs, emphasizing the crucial role of the hippocampus in WM tasks. This suggests a new hypothesis: the hippocampus changes its functional state from encoding to retrieval during the presence of SWRs.
\end{abstract}