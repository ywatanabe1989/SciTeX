\begin{abstract}
\pdfbookmark[1]{Abstract}{abstract}
Working memory (WM) is critical to various cognitive functions, yet its neural mechanisms predominantly remain undefined. A burgeoning area of study examines the contribution of the hippocampus and sharp wave-ripple complexes (SWRs) – ephemeral, synchronized neural events in the hippocampus – to memory consolidation and retrieval. However, their role in WM tasks remains obscure. Recent studies suggest a concurrent function of multiunit activity patterns in the hippocampus with SWRs, showing distinctive dynamics during WM tasks. We analyzed an electroencephalogram dataset from the medial temporal lobe (MTL) obtained from nine epilepsy patients during an eight-second Sternberg task. Low-dimensional neural representations, referred to as 'trajectories', were isolated from the MTL using Gaussian-process factor analysis during the WM task. The analysis reveals significant differences in these neural trajectories within the hippocampus as compared to those in the entorhinal cortex and the amygdala. Moreover, the variance in trajectories between the encoding and retrieval phases appears to be dependant on memory load. Notably, hippocampal trajectories shift during the retrieval phase, indicating task-dependent transitions between encoding and retrieval states, evident during both baseline and SWR events. These transitions synchronise with the occurrence of SWRs, underscoring the crucial role of the hippocampus in WM tasks, and proposes a new hypothesis: the functional state of the hippocampus toggles from encoding to retrieval during SWR presence.
\end{abstract}