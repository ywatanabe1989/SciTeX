\begin{abstract}
\pdfbookmark[1]{Abstract}{abstract}
Working memory (WM) serves as a critical cornerstone in a multitude of cognitive functions; yet, the elaborate neural mechanisms underpinning its functionality remain incompletely understood. Specifically, while both the hippocampus and sharp-wave ripple complexes (SWRs) --- rapid, coordinated neural occurrences within the hippocampus --- are recognized for their roles in memory consolidation and retrieval, their involvement in WM tasks persists as somewhat equivocal. Our present research theorizes that the multiunit activity patterns within the hippocampus work synergistically with SWRs, thereby demonstrating distinctive dynamism during WM tasks. Our investigation involved an in-depth analysis of a dataset derived from intracranial electroencephalogram recordings acquired from the medial temporal lobe (MTL) of nine individuals with epilepsy performing an eight-second Sternberg task. We employed Gaussian-process factor analysis to discern low-dimensional neural representations, or 'trajectories,' within the MTL territories during the WM task. Our findings revealed that the neural trajectory exhibited the most significant variations in the hippocampus when compared to the entorhinal cortex and amygdala. Moreover, divergence in the trajectories identified between encoding and retrieval phases were memory load-dependent. Importantly, hippocampal trajectories oscillated during the retrieval phase, displaying task-dependent transitions between encoding and retrieval states, inclusive of baseline and SWR episodes. These oscillations shifted from encoding to retrieval states in accordance with SWRs. Hence, these results highlight the critical role of the hippocampus in performing WM tasks and propose a persuasive hypothesis for subsequent investigation: the functional state of the hippocampus transitions from encoding to retrieval during SWRs.
\end{abstract}