\section{Discussion}
We hypothesize in this study that hippocampal neurons form unique neural trajectories (NTs) in low-dimensional spaces, primarily during sharp-wave ripple (SWR) periods, while undertaking a working memory (WM) task in humans. Initially, multi-unit spikes in medial temporal lobe (MTL) regions were projected onto three-dimensional spaces during a Sternberg task using Gaussian process factor analysis (GPFA) (Figure~\ref{fig:01}D--E \& Figure~\ref{fig:02}A). The NT distances across WM phases ($\mathrm{\lVert g_{F}g_{E} \rVert}$, $\mathrm{\lVert g_{F}g_{M} \rVert}$, $\mathrm{\lVert g_{F}g_{R} \rVert}$, $\mathrm{\lVert g_{E}g_{M} \rVert}$, $\mathrm{\lVert g_{E}g_{R} \rVert}$, and $\mathrm{\lVert g_{M}g_{R} \rVert}$) were notably larger in the hippocampus than in the entorhinal cortex (EC) and amygdala (Figure~\ref{fig:02}C--E). This is indicative of dynamic and responsive neural activity in the hippocampus during the WM task. Moreover, in the hippocampus, the NT distance between the encoding and retrieval phases ($\mathrm{\lVert g_{F}g_{E} \rVert}$) showed positive correlation with memory load (Figure~\ref{fig:03}C--D), suggesting active WM processing. The hippocampal neural NTs transiently expanded during SWRs (Figure~\ref{fig:05}), and alternated between encoding and retrieval states, transitioning from encoding to retrieval during SWR events (Figure~\ref{fig:07}). These findings elucidate aspects of hippocampal neural activity during a WM task in humans and provide new insights into SWRs as a state-switching element in hippocampal neural states.

The NT distance across the phases was significantly longer in the hippocampus than in the EC and amygdala, even when considering the distance from $O$ in these regions (Figure~\ref{fig:02}C--E). This validates hippocampal involvement in the WM task, aligning with previous studies indicating persistent hippocampal firing during the maintenance phase \cite{boran_persistent_2019} \cite{kaminski_persistently_2017} \cite{kornblith_persistent_2017} \cite{faraut_dataset_2018}. In this study, application of GPFA to multi-unit activity during one-second levels of the WM task revealed that the neural NT in low-dimensional space exhibited a memory-load dependency between the encoding and retrieval phases, denoted as $\mathrm{\lVert g_{E}g_{R} \rVert}$ (Figure~\ref{fig:03}). This supports the association of the hippocampus with WM processing.

The analysis focused on putative CA1 regions (Figure~\ref{fig:04}), resting on several factors. The focus on these regions resulted from observations that SWRs synchronize with interneuron and pyramidal neuron spike bursts \cite{buzsaki_two-stage_1989} \cite{quyen_cell_2008} \cite{royer_control_2012} \cite{hajos_input-output_2013}, potentially within a 50 $\mu$m radius of the recording site \cite{schomburg_spiking_2012}. Also, we identified increased SWR incidence during the first 0--400 ms of the retrieval phase (Figure~\ref{fig:04}D), a finding that aligns with previous reports of elevated SWR occurrence preceding spontaneous verbal recall \cite{norman_hippocampal_2019} \cite{norman_hippocampal_2021}, bolstering our results under a triggered retrieval condition. The observed log-normal distributions of both SWR duration and ripple band peak amplitude in this study (Figure~\ref{fig:04}C \& E) align with prevailing consensus in this field \cite{liu_consensus_2022}. Thus, our decision to confine recording sites to putative CA1 regions likely improved the precision, or true positive rate, of SWR detection. Although, the NT distance increase from $O$ during SWRs (Figure~\ref{fig:05}) might be artificially biased towards higher values due to channel selection, this potential bias does not substantially alter our main findings.

Interestingly, during the retrieval phase, NT directions alternated between encoding and retrieval states during both baseline and SWR periods in a task-dependent way (Figure~\ref{fig:07}C \& D). Additionally, the balance of this fluctuation transitioned from encoding to retrieval state during SWR events (Figure~\ref{fig:07} E \& F). The results align with previous studies on the role of SWR in memory retrieval \cite{norman_hippocampal_2019} \cite{norman_hippocampal_2021}. Our findings suggest (i) neuronal oscillation between encoding and retrieval states occurs during a WM task, and (ii) SWR events serve as indicators of the transition in hippocampal neural states from encoding to retrieval during a WM task.

Notably, our study observed differences specific to the WM-task type between encoding- and retrieval-SWRs (Figure~\ref{fig:07}E--F). Opposing movements of encoding-SWR (eSWR) and retrieval-SWR (rSWR) were not visible in the Match IN task but were apparent in the Mismatch OUT task. The memory engram theory \cite{liu_optogenetic_2012} might illuminate these observations: Match In task exposed participants to previously shown letters, while Mismatch OUT task introduced a new letter not present in the encoding phase. This provides an explanation emphasizing the crucial role of SWR in human cognitive processes.

In conclusion, this study demonstrates that during a WM task in humans, hippocampal activity oscillates between encoding and retrieval states, transitioning uniquely from encoding to retrieval during SWR events. These findings offer fresh insights into the neural correlates and function of working memory within the hippocampus.
\label{sec:discussion}