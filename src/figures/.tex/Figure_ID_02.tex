        \clearpage
        \begin{figure*}[ht]
            \pdfbookmark[2]{ID 02}{figure_id_02}
        	\centering
            \includegraphics[width=]{./src/figures/.png/Figure_ID_02.png}
        	\caption{\textbf{
State-dependent hippocampal neural trajectory
}
\smallskip
\\
\textbf{\textit{A.}} The figure illustrates the neural trajectory in the first three dimensions computed using the Gaussian Process Factor Analysis (GPFA). Each smaller dot represents coordinates of 50-ms neural trajectory bins, whereas larger dots outlined in \textit{black} symbolize geometric medians of consecutive phases in the Sternberg working memory task: fixation (\textit{gray}), encoding (\textit{blue}), maintenance (\textit{green}), and retrieval (\textit{red})\cite{yu_gaussian-process_2009}. \textbf{\textit{B.}} The graph reveals the log-likelihood of GPFA models in relation to the number of dimensions employed for embedding multi-unit spikes in medial temporal lobe (MTL) regions. Significantly, the dimensionality's optimal value was identified as three, determined using the elbow method\cite{virtanen_scipy_2020}. \textbf{\textit{C.}} This portion maps the distance of the neural trajectory from the origin ($O$) for the hippocampus (Hipp.), entorhinal cortex (EC), and amygdala (Amy.), and plots it against time from the probe's initiation \cite{boran_dataset_2020}. \textbf{\textit{D.}} The subsequent graph showcases the trajectory's distance from $O$ across MTL regions, with the hippocampus demonstrating the greatest distance, followed by the EC and the Amygdala\cite{fernandez-ruiz_long-duration_2019}. \textbf{\textit{E.}} The final representation indicates inter-phase trajectory distances within the MTL regions\cite{liu_consensus_2022}.
Abbreviations:
}
        	\label{fig:02}
        \end{figure*}
