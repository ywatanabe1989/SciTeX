        \clearpage
        \begin{figure*}[ht]
            \pdfbookmark[2]{ID 02}{figure_id_02}
        	\centering
            \includegraphics[width=]{./src/figures/.png/Figure_ID_02.png}
        	\caption{\textbf{
State-Dependent Neural Trajectories of Hippocampal Neurons
}
\smallskip
\\
\textbf{\textit{A.}} Neural trajectories (NTs) are represented as a point cloud within the first three-dimensional factors derived from Gaussian Process Factor Analysis (GPFA) \cite{yu_gaussian-process_2009}. The smaller dots represent 50-ms NT bins, while the larger dots with black edges denote the geometric medians for each phase of the Sternberg working memory task. These phases include fixation ($\mathrm{\lVert g_{F} \rVert}$, gray), encoding ($\mathrm{\lVert g_{E} \rVert}$, blue), maintenance ($\mathrm{\lVert g_{M} \rVert}$, green), and retrieval ($\mathrm{\lVert g_{R} \rVert}$, red). \textbf{\textit{B.}} The figure shows the log-likelihood of GPFA models against the number of dimensions used to embed multi-unit spikes found in the medial temporal lobe (MTL) regions. Specifically, the elbow method identified three as the optimal dimension. \textbf{\textit{C.}} This panel depicts the distance of the NTs from the origin ($O$) for the hippocampus (Hipp.), entorhinal cortex (EC), and amygdala (Amy.), plotted against the elapsed time since probe onset. \textbf{\textit{D.}} Here, the NT distance from $O$ within the MTL regions is displayed. The greatest distance is in the hippocampus, followed by the EC and the Amygdala. \textbf{\textit{E.}} The box plot shows the inter-phase NT distances within the MTL regions.
}
        	\label{fig:02}
        \end{figure*}
