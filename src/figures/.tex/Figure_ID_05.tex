        \clearpage
        \begin{figure*}[ht]
            \pdfbookmark[2]{ID 05}{figure_id_05}
        	\centering
            \includegraphics[width=1\textwidth]{./src/figures/.png/Figure_ID_05.png}
        	\caption{\textbf{
Transient Alterations in Neural Trajectory During SWR
}
\smallskip
\\
\textbf{\textit{A.}} Represents the distance from the origin ($O$) of the peri-sharp-wave-ripple (SWR) trajectory expressed as the mean \textpm a 95\% confidence interval, which may not be visible owing to its narrow range \cite{girardeau_selective_2009,norman_hippocampal_2019,buzsaki_hippocampal_2015}. \textbf{\textit{B.}} Illustrates the distance from the origin ($O$) during the pre-, mid-, and post-SWR periods (*\textit{p} $<$ 0.05, **\textit{p} $<$ 0.01, ***\textit{p} $<$ 0.001; according to the Brunner--Munzel test \cite{boran_persistent_2019}). Abbreviations detailed as: SWR, sharp-wave ripple events; eSWR, SWR occurring in the encoding phase; rSWR, SWR happening during the retrieval phase; SWR$^+$, SWR event; SWR$^-$, control events for SWR$^+$; pre-, mid-, or post-SWR, the time intervals from $-800$ to $-250$ ms, from $-250$ to $+250$ ms, or from $+250$ to $+800$ ms, each relative to the SWR center.
}
% width=1\textwidth
        	\label{fig:05}
        \end{figure*}
