        \clearpage
        \begin{figure*}[ht]
            \pdfbookmark[2]{ID 03}{figure_id_03}
        	\centering
            \includegraphics[width=1\textwidth]{./src/figures/png/Figure_ID_03.png}
        	\caption{\textbf{
Positive Correlation between Memory Load and Neural Trajectory Distance in the Hippocampus between Encoding and Retrieval Phases
}
\smallskip
\\
\textbf{\textit{A.}} The relationship between set size (number of letters to be encoded) and accuracy in the working memory task (coefficient = $-0.20$, ***\textit{p} $<$ 0.001). \textbf{\textit{B.}} The correlation between set size and response time (coefficient = 0.23, ***\textit{p} $<$ 0.001). \textbf{\textit{C.}} The correlation between set size on the inter-phase distances between the encoding and retrieval phases ($\lVert \mathrm{g_{E}g_{R}} \rVert$) (correlation coefficient = 0.05, ***\textit{p} $<$ 0.001). \textbf{\textit{D.}} Experimental observations of correlations between set size and the following parameters: accuracy, response time, $\log_{10}{\lVert \mathrm{g_{F}g_{E}} \rVert}$, $\log_{10}{\lVert \mathrm{g_{F}g_{M}} \rVert}$, $\log_{10}{\lVert \mathrm{g_{F}g_{R}} \rVert}$, $\log_{10}{\lVert \mathrm{g_{E}g_{M}} \rVert}$, $\log_{10}{\lVert \mathrm{g_{E}g_{R}} \rVert}$, and $\log_{10}{\lVert \mathrm{g_{M}g_{R}} \rVert}$ represented by \textit{red} dots. The kernel density plots (\textit{gray}) illustrate the corresponding shuffled surrogate with set size (\textit{n} = 1,000) (***\textit{p}s $<$ 0.001).
}
% width=1\textwidth
        	\label{fig:03}
        \end{figure*}
