\caption{\textbf{
State-dependent Hippocampal Neural Trajectory
}
\smallskip
\\
\textbf{\textit{A.}} This figure illustrates the neural trajectory in the first three dimensions, which are computed using the Gaussian Process Factor Analysis (GPFA). Each smaller dot represents the coordinates of a 50-ms neural trajectory bin, while larger dots marked in \textit{black} signify the geometric medians of successive phases in the Sternberg working memory task. These phases include fixation (\textit{gray}), encoding (\textit{blue}), maintenance (\textit{green}), and retrieval (\textit{red})\cite{yu_gaussian-process_2009}. \textbf{\textit{B.}} The graph displays the log-likelihood of GPFA models relative to the number of dimensions used for embedding multi-unit spikes in medial temporal lobe (MTL) regions. Importantly, the optimal value of dimensionality was identified as three, using the elbow method\cite{virtanen_scipy_2020}. \textbf{\textit{C.}} This section maps the distance between the neural trajectory and the origin ($O$) for the hippocampus (Hipp.), entorhinal cortex (EC), and amygdala (Amy.), and plots it against time from the commencement of the probe \cite{boran_dataset_2020}. \textbf{\textit{D.}} The following graph highlights the trajectory's distance from $O$ across MTL regions, with the hippocampus displaying the greatest distance, followed by the EC and Amygdala\cite{fernandez-ruiz_long-duration_2019}. \textbf{\textit{E.}} The final representation indicates the inter-phase trajectory distances within the MTL regions\cite{liu_consensus_2022}.
Abbreviations:
}