\section{Methods}
\subsection{Dataset}
The dataset employed in this study was publicly available and consisted of nine epilepsy patients performing a modified Sternberg task \cite{boran_dataset_2020}. This task encompassed four phases: fixation (1s), encoding (2s), maintenance (3s), and retrieval (2s). During the encoding phase, participants were presented with four, six, or eight alphabet letters, defined as the set size. They [FIXME>]were required to[<FIXME] ascertain whether a probe letter shown during the retrieval phase had appeared before (the correct choice for the Match IN task) or not (the correct choice for the Mismatch OUT task). Intracranial electroencephalography (iEEG) signals were captured with a 32 kHz sampling rate within a 0.5--5,000 Hz frequency range, using depth electrodes in medial temporal lobe (MTL) regions: the anterior head of the left and right hippocampus (AHL and AHR), the posterior body of the hippocampus (PHL and PHR), the entorhinal cortex (ECL and ECR), and the amygdala (AL and AR), as depicted in Figure~\ref{fig:01}A and Table~\ref{tab:01}. The iEEG signals were subsequently downsampled to 2 kHz. Correlations between variables such as set size and correct rate were examined (Figure~\ref{fig:s01}S1). Multiunit spike timings were determined via a spike sorting algorithm \cite{niediek_reliable_2016} using the Combinato package (\url{https://github.com/jniediek/combinato})(Figure~\ref{fig:01}C).

\subsection{Calculation of neural trajectories using GPFA}
Neural trajectories, also referred to as 'factors' (Figure~\ref{fig:01}D), in the hippocampus, EC, and amygdala were determined using GPFA \cite{yu_gaussian-process_2009} applied to the multiunit activity data for each session, performed with the elephant package (\url{https://elephant.readthedocs.io/en/latest/reference/gpfa.html}). The bin size was set to 50 ms, without overlaps. Each factor was z-normalized across all sessions, and the Euclidean distance from the origin ($O$) was then computed (Figure~\ref{fig:01}E).

For each trajectory within a region such as AHL, geometric medians ($\mathrm{g_{F}}$ for fixation, $\mathrm{g_{E}}$ for encoding, $\mathrm{g_{M}}$ for maintenance, and $\mathrm{g_{R}}$ for retrieval phase) were calculated by determining the median coordinates of the trajectory during the four phases (Figure~\ref{fig:01}D). An optimal GPFA dimensionality was found to be three using the elbow method obtained by examining the log-likelihood values through a three-fold cross-validation approach (Figure~\ref{fig:02}B).

\subsection{Identifying SWR candidates from hippocampal regions}
Potential SWR events within the hippocampus were detected using a widely used method \cite{liu_consensus_2022}. LFP signals from a region of interest (ROI) like AHL, were re-referenced by deducting the averaged signal from locations outside the ROI (for instance, AHR, PHL, PHR, ECL, ECR, AL, and AR) (see Figure~\ref{fig:01}A). The re-referenced LFP signals were then filtered with a ripple-band filter (80--140 Hz) to determine SWR candidates, marked as $\textrm{SWR}^+$ candidates (see Figure~\ref{fig:01}B). SWR detection was carried out using a published tool (\url{https://github.com/Eden-Kramer-Lab/ripple_detection}) \cite{kay_hippocampal_2016}, with the bandpass range adjusted to 80--140 Hz for humans \cite{norman_hippocampal_2019} \cite{norman_hippocampal_2021}, unlike the initial 150--250 Hz range typically applied to rodents.

Control events for $\textrm{SWR}^+$ candidates, labeled as $\textrm{SWR}^-$ candidates, were detected by randomly shuffling the timestamps of $\textrm{SWR}^+$ candidates across all trials and subjects. The resulting $\textrm{SWR}^+/\textrm{SWR}^-$ candidates were then visually inspected (Figure~\ref{fig:01}).

\subsection{Defining SWRs from putative hippocampal CA1 regions}
SWRs were differentiated from SWR candidates in putative CA1 regions. These regions were initially defined as follows: $\textrm{SWR}^+/\textrm{SWR}^-$ candidates in the hippocampus were projected into a two-dimensional space based on overlapping spike counts per unit using a supervised method, UMAP (Uniform Manifold Approximation and Projection) \cite{mcinnes_umap_2018} (Figure~\ref{fig:04}A). Clustering validation was performed by calculating the silhouette score \cite{rousseeuw_silhouettes_1987} from clustered samples (Table~\ref{tab:02}). Regions in the hippocampus, which scored above 0.6 on average across sessions (75th percentile) (Figure~\ref{fig:04}B), were identified as putative CA1 regions, resulting in the identification of five electrode positions from five patients (Table~\ref{tab:03}).

$\textrm{SWR}^+/\textrm{SWR}^-$ candidates in these predetermined CA1 regions were categorized as $\textrm{SWR}^+/\textrm{SWR}^-$, and thus they no longer retained their candidate status. The duration and ripple band peak amplitude of SWRs were found to follow log-normal distributions (Figure~\ref{fig:04}C \& E). Each time period of SWR was partitioned relative to the time from the SWR center into pre- (at $-800$ to $-300$ ms from the SWR center), mid- (at $-250$ to $+250$ ms), and post-SWR (at $+300$ to $+800$ ms) times.

\subsection{Statistical evaluation}
Both the Brunner--Munzel test and the Kruskal-Wallis test were executed using the SciPy package in Python \cite{virtanen_scipy_2020}. Correlational analysis was conducted by determining the rank of the observed correlation coefficient within its associated set-size-shuffled surrogate using a customized Python script. The bootstrap test was implemented with an in-house Python script.

\label{sec:methods}