\section{Methods}
\subsection{Dataset}
This study utilized a publicly available dataset that comprises nine epilepsy patients performing a modified Sternberg task \cite{boran_dataset_2020}. This task includes four phases: fixation (1s), encoding (2s), maintenance (3s), and retrieval (2s). During the encoding phase, participants were presented with a set of four, six, or eight alphabet letters. Their task was to determine whether a probe letter displayed during the retrieval phase was previously presented (the accurate response in a Match IN task) or not (the accurate response for a Mismatch OUT task). Intracranial electroencephalography (iEEG) signals were collected with a 32 kHz sampling rate in the 0.5--5,000 Hz frequency range, using depth electrodes targeting medial temporal lobe (MTL) regions: the anterior head of the left and right hippocampus (AHL and AHR), the posterior body of the hippocampus (PHL and PHR), the entorhinal cortex (ECL and ECR), and the amygdala (AL and AR), as illustrated in Figure~\ref{fig:01}A and Table~\ref{tab:01}. The iEEG signals were later downsampled to 2 kHz. Correlations between variables like set size and correct rate were examined (Figure~\ref{fig:s01}S1). Multiunit spike timings were identified using a spike sorting algorithm \cite{niediek_reliable_2016} and the Combinato package (\url{https://github.com/jniediek/combinato})(Figure~\ref{fig:01}C).

\subsection{Calculation of neural trajectories using GPFA}
Neural trajectories, also known as 'factors', in the hippocampus, EC, and amygdala were calculated using GPFA \cite{yu_gaussian-process_2009} applied to the multiunit activity data for each session, implemented with the elephant package (\url{https://elephant.readthedocs.io/en/latest/reference/gpfa.html}). The bin size was set to 50 ms, without overlaps. Each factor was z-normalized across all sessions, after which the Euclidean distance from the origin ($O$) was computed. For each trajectory within a region such as AHL, geometric medians ($\mathrm{g_{F}}$ for fixation, $\mathrm{g_{E}}$ for encoding, $\mathrm{g_{M}}$ for maintenance, and $\mathrm{g_{R}}$ for retrieval phase) were calculated by determining the median coordinates of the trajectory during the four phases. Optimal GPFA dimensionality was established as three using the elbow method obtained by examining the log-likelihood values via a three-fold cross-validation approach (Figure~\ref{fig:02}B).

\subsection{Identifying SWR candidates from hippocampal regions}
Potential SWR events in the hippocampus were identified using a widely recognized method \cite{liu_consensus_2022}. LFP signals from a region of interest (ROI), such as AHL, were re-referenced by subtracting the average signal from locations outside the ROI (for instance, AHR, PHL, PHR, ECL, ECR, AL, and AR). The re-referenced LFP signals were then filtered with a ripple-band filter (80--140 Hz) to identify SWR candidates, denoted as $\textrm{SWR}^+$ candidates. SWR detection was conducted using a published tool (\url{https://github.com/Eden-Kramer-Lab/ripple_detection}) \cite{kay_hippocampal_2016}, with the bandpass range adjusted to 80--140 Hz for humans \cite{norman_hippocampal_2019} \cite{norman_hippocampal_2021}, contrasting with the initial 150--250 Hz range typically applied to rodents. Control events for $\textrm{SWR}^+$ candidates, labeled as $\textrm{SWR}^-$ candidates, were detected by randomly shuffling the timestamps of $\textrm{SWR}^+$ candidates across all trials and subjects. The resulting $\textrm{SWR}^+/\textrm{SWR}^-$ candidates were subsequently visually inspected.

\subsection{Defining SWRs from putative hippocampal CA1 regions}
Potential SWRs were distinguished from SWR candidates in presumptive CA1 (cornu Ammonis 1) regions. These regions were initially defined as follows: $\textrm{SWR}^+/\textrm{SWR}^-$ candidates in the hippocampus were projected into a two-dimensional space based on overlapping spike counts per unit using a supervised approach, UMAP (Uniform Manifold Approximation and Projection) \cite{mcinnes_umap_2018}. Clustering validation was conducted by calculating the silhouette score \cite{rousseeuw_silhouettes_1987} from clustered samples. Regions in the hippocampus, which scored on average above 0.6 across sessions (75th percentile), were identified as putative CA1 areas, resulting in the identification of five electrode positions from five patients. $\textrm{SWR}^+/\textrm{SWR}^-$ candidates in these predetermined CA1 areas were classified as $\textrm{SWR}^+/\textrm{SWR}^-$, thereby relinquishing their candidate status. The duration and ripple band peak amplitude of SWRs were found to follow log-normal distributions. Each SWR period was segmented relative to the time from the SWR center into pre- (at $-800$ to $-300$ ms from the SWR center), mid- (at $-250$ to $+250$ ms), and post-SWR (at $+300$ to $+800$ ms) times.

\subsection{Statistical evaluation}
Both the Brunner--Munzel test and the Kruskal-Wallis test were administered using the SciPy package in Python \cite{virtanen_scipy_2020}. Correlational analysis was performed by determining the rank of the observed correlation coefficient within its associated set-size-shuffled surrogate using a customized Python script. The bootstrap test was performed with an in-house Python script.
\label{sec:methods}